\documentclass[PICOAPC.tex]{subfiles}

%\newcolumntype{L}[1]{>{\raggedright\let\newline\\\arraybackslash\hspace{0pt}}m{#1}}
%\newcolumntype{K}[1]{>{\raggedright\centering\arraybackslash}m{#1}}

\begin{document}

Properly modeling, engineering for, and controlling systematic effects are key for the success of any experimental endeavor striving to achieve $\sigma(r) \lesssim 1 \times 10^{-3}$. Based on extensive community experience with both hardware and analysis of data we make the following points.  \\
$\bullet$ \hspace{0.1in}  Relative to other platforms, a space-based mission provides the most thermally stable platform, and thus the prerequisite for improved control of systematic effects. PICO's orbit at L2 is among the most thermally stable of possible orbits. \\
$\bullet$ \hspace{0.1in} PICO's sky scan pattern gives strong data redundancy, which enables numerous cross-checks. Each of the 12,996 detectors makes independent maps of the $I,\,Q$, and $U$ Stokes parameters enabling many comparisons within and across frequency bands, within and across sections of the focal plane, and within and across bolometers that have either the same or different polarization sensitivities. Half the sky is scanned every two weeks, and the entire sky is scanned in 6 months. Thus combinations of maps constructed at different times during of the mission will be differenced to search for residual time-dependent systematic effects. \\
$\bullet$ \hspace{0.1in}  The scan pattern gives almost continuous scans of planets and large amplitude ($\geq 4$~mK) CMB dipole signals~\citep{picoweb_dipole}. These features result in continuous, high \ac{SNR} calibration and antenna-pattern characterization. In comparison, \planck\ observed each of the planets with only a 6 month cadence and had nearly 100~days/year during which the dipole calibration signals were below 4~mK, at times dipping below 1~mK. \\
$\bullet$ \hspace{0.1in}  We showed that two of the highest priority systematic effects can be controlled to levels that are small compared to requirements. More analysis and planning is required to address systematic uncertainties arising from the far-sidelobe response of the telescope. 

We direct the reader to the mission study report for more details on our work on systematic effects for PICO~\citep{pico_report}.

\end{document}


