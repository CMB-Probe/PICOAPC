\documentclass[PICOAPC.tex]{subfiles}

\begin{document}

\definecolor{mygray}{gray}{0.6}

PICO was designed to respond to requirements posed by the seven \ac{SOs} listed in Table~\ref{tab:STM}. It will also generate a rich catalog of hundreds of thousands of new sources, consisting of proto-clusters, strongly lensed galaxies, and polarized radio and dusty galaxies. An abundance of information about galaxy and cluster evolution, dark matter, the physics of jets of active galactic nuclei, and magnetic fields of dusty galaxies will be stored in this catalog (Table~\ref{tab:STM2}). The catalog will be mined in future years through subsequent analysis and follow-up observations. 
 \begin{table}[h]
\caption{\textbf{Legacy Surveys } }\label{tab:STM2}
\footnotesize
\vspace{-0.1in}
\begin{tabular*}{\textwidth}{@{}l@{\extracolsep{\fill}}ll@{}}
\noalign{\vskip 2mm}
\hline
\noalign{\vskip 1mm}
\hline
\noalign{\vskip 2mm}    
% Header line 1
{\bf \hfil Catalog\hfil}&
{\bf \hfil Impact\hfil}&
{\bf \hfil Science\hfil}\\
% Line 1
\noalign{\vskip 2mm}    
\hline
\noalign{\vskip 1mm}    

\parbox[t]{0.8in}{Strongly\\ lensed galaxies}&
\parbox[t]{2.55in}{Discover 4500$^a$ strongly lensed and highly magnified dusty galaxies across redshift. 
\vspace{1mm}
{\color{mygray}\hrule}
\vspace{1mm}
Current knowledge: 13 sources confirmed in \planck\ data; a few hundred candidates in \textit{Herschel}, SPT and ACT data.}&
\parbox[t]{2.7in}{Gain information about the physics governing early, $z\simeq5$, galaxy evolution, taking advantage of magnification and extra resolution enabled by gravitational lensing;  learn about dark matter sub-structure in the lensing galaxies.}\\
% Line 3
\noalign{\vskip 1mm}    
\cline{1-3}
\noalign{\vskip 1mm}    

\parbox[t]{0.8in}{Proto-clusters}&
\parbox[t]{2.55in}{Discover 50,000$^{a}$ mm/sub-mm proto-clusters distributed over the sky out to $z\sim4.5$.  
\vspace{1mm}
{\color{mygray}\hrule}
\vspace{1mm}
Current knowledge: \planck\ + ACT/SPT data expected to yield a few tens.}&
\parbox[t]{2.7in}{Probe the earliest phases of cluster evolution, well beyond the reach of other instruments; test the formation history of the most massive virialized halos; investigate galaxy evolution in dense environments.}\\
% Line 2
\noalign{\vskip 1mm}    
\hline
\noalign{\vskip 1mm}    

\parbox[t]{0.8in}{Nearby galaxies}&
\parbox[t]{2.55in}{Detect 30,000 galaxies at $z\simlt 0.1$ at frequencies above 300~GHz.  
\vspace{1mm}
{\color{mygray}\hrule}
\vspace{1mm}
Current knowledge: 3400 (280) source candidates in the \planck\ 857 (353)~GHz  band. }&
\parbox[t]{2.7in}{Using frequencies that match cold ($15-25$~K) dust emission, give its spectral energy distribution as a function of galaxy properties to enable correlations with star-formation activity.} \\
\noalign{\vskip 1mm}
\hline 
\noalign{\vskip 1mm}

\parbox[t]{0.8in}{Polarized point\\ sources}&
\parbox[t]{2.55in}{Detect 2000$^{b}$ radio and several thousand dusty galaxies in polarization. 
\vspace{1mm}
{\color{mygray}\hrule}
\vspace{1mm}
Current knowledge:  about 200 radio sources up to 100~GHz; one polarization measurement of a dusty galaxy. }&
\parbox[t]{2.7in}{Study the physics of jets of extragalactic sources, close to their active nuclei; determine the large-scale structure of magnetic fields in dusty galaxies; determine the importance of polarized sources as a foreground for CMB polarization science.}\\
\noalign{\vskip 1mm}
\hline
\noalign{\vskip 1mm}

\parbox[t]{0.8in}{Cosmic infrared \\ background}&
\parbox[t]{2.55in}{Provide eight maps of the anisotropy from dusty star-forming galaxies for frequencies $\nu>200$~GHz, and with 1\arcmin\ resolution at 800~GHz.
\vspace{1mm}
{\color{mygray}\hrule}
\vspace{1mm}
Current knowledge:  Three \planck\ (higher noise) maps between 300 and 900~GHz with 5\arcmin\,~resolution. }&
\parbox[t]{2.7in}{Improve constraints on the parameters describing universal star-formation history. Construct a tracer of large-scale structure for CMB de-lensing. Cross-correlate with galaxy surveys and CMB lensing map.}\\
\noalign{\vskip 1mm}
\hline
\noalign{\vskip 1mm}

\end{tabular*}
{\footnotesize
$^a$ Confusion (not noise) limited\qquad
$^b$ Noise and confusion limited }
\end{table}

\end{document}

%%%%%%%%%%%%%%%%%%%%%%%%%%%%%%%
