\documentclass[PICOAPC.tex]{subfiles}
\newcommand\pdeg{.\!\!\degree}
\newcommand\parcm{.\!\!'}


\section{Organization, Partnerships, and Current Status}
\label{sec:organization} %6

%\subsection{PICO Study Participants}
%\label{sec:study_participants} %6.1

PICO is the result of an 18-months mission study funded by NASA (total grant = \$150,000). The study was open to the entire mm/sub-mm community. Seven working groups were led by members of PICO's Executive Committee, which had a telephone conference weekly, led by the PI. A three-member steering committee, composed of two experimentalists experienced with CMB space missions, and a senior theorist gave occasional advice to the PI. More than 60 scientists, international- and US-based, participated in-person in each of two community workshops (November 2017 and May 2018). The study report has been submitted by NASA to the decadal panel, and it is available on the arXiv and on the PICO website~\cite{pico_report, picoweb}. It has contributions from 82 authors, and has been endorsed by additional 131 members of the community. 

%The PICO engineering concept definition package including the total cost was generated by Team~X.\footnote{\label{teamx} Team~X is JPL's concurrent design facility.} The Team~X study was supported by inputs from a JPL engineering team and Lockheed Martin.

The PICO team designed an entirely US-based mission, so that the full cost of the mission can be assessed. We excluded contributions by other space agencies, despite expression of interest by international scientists. The PICO concept has wide support in the international community. If the mission is selected to proceed, a path that would be scientifically and financially optimal relative to other options, it is reasonable to expect that international partners would participate and thus reduce the US cost of the mission. 




%\end{document}
