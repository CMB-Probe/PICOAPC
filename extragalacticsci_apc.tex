\documentclass[PICOAPC.tex]{subfiles}

\begin{document}

 
$\bullet$ {\bf The Formation of the First Luminous Sources} \hspace{0.1in} \label{sec:luminoussources}  A few hundred million years after the Big Bang, the neutral hydrogen gas permeating the Universe was reionized by photons emitted by the first luminous sources to have formed.  The nature of these sources and the exact history of this epoch are key missing links in our understanding of structure formation.  With full sky coverage, multiple frequency bands, and ample sensitivity to remove foregrounds, PICO is uniquely suited to make the low-$\ell$ $EE$-spectrum measurements and reach cosmic-variance-limited precision with $\sigma(\tau)=0.002$, settling some of these questions and significantly constraining the others. Data from PICO's frequency bands above 400~GHz -- which have better than 2~arcmin resolution  -- will be used to provide clean maps for higher resolution ground-based instruments that can reconstruct the patchy $\tau$ field. No other experiment can provide these data. \\
$\bullet$ {\bf Probing the Evolution of Structures via Gravitational Lensing and Cluster Counts} \hspace{0.1in} \label{sec:gravitationallensing}   The amplitude of linear fluctuations as a function of redshift, parameterized by $\sigma_8(z)$, is a sensitive probe of physical processes affecting growth of structures in the Universe. \ac{CMB} photons are affected by, and thus probe, $\sigma_{8}(z)$ as they traverse the entire Universe. The PICO sub-percent constraints on $\sigma_8(z)$, obtained through measurements of gravitational lensing and independently through using cluster counts, will translate to constraints on dark energy, models of modified gravity, baryonic feedback process, and limits on the particle content of the Universe. 

{\bf -- Gravitational Lensing} \hspace{0.1in} \label{lensing} Matter between us and the last-scattering surface deflects the path of photons through gravitational lensing, imprinting the three-dimensional matter distribution across the volume of the Universe onto the CMB maps. The specific quantity being mapped by these data is the projected gravitational potential $\phi$ that is lensing the photons. With \ac{SNR} of more than 560, the PICO $C_{L}^{\phi \phi}$ angular power spectrum is the highest of any foreseeable CMB experiment in the range $2 \leq L \lesssim 1500$. 
PICO's $\phi$ map will be a key ingredient in the delensing process that improves constraints on $r$, in extracting neutrino mass constraints, in constraining shear biases for LSST and WFIRST~\cite{schaan}, and in measuring $\sigma_{8}(z)$ in multiple redshift bins with sub-percent accuracy~\citep{pico_report}. 

{\bf -- Cluster Counts} \hspace{0.1in} \label{clusters}  
The distribution of galaxy clusters over redshift is one consequence of the evolution of structures and is thus a sensitive measure of $\sigma_{8}(z)$. We forecast that PICO will find $\sim$150,000 galaxy clusters, assuming the cosmological parameters from \planck\  and using the 70\% of sky not obscured by the Milky Way.  Information provided by the high frequency bands will mitigate the potential reduction in detection efficiency due to dust emission by cluster members~\citep{melin_2018}. This catalog will provide $\sigma_{8}$ with sub-percent precision for $0.5 < z < 2$, and a neutrino mass constraint $\sigma(\sum m_{\nu}) = 14$~meV that is independent from the one coming from the CMB lensing measurements. A significant fraction of the PICO-detected clusters will also be detected by eROSITA, giving an exceptional catalog of multi-wavelength observations for detailed studies of cluster astrophysics. 


$\bullet$ {\bf Constraining Feedback Processes through the Sunyaev--Zeldovich Effect} \hspace{0.1in} \label{sec:sz}
About 6\% of CMB photons are Thomson-scattered by free electrons in the \ac{IGM} and \ac{ICM}, and a fraction of these are responsible for the thermal and kinetic Sunyaev--Zeldovich effects (tSZ and kSZ)~\citep{zeldovich69,SZ1972}. The amplitude of the tSZ is proportional to the integrated electron pressure along the line of sight, and it thus contains information about the thermodynamic properties of the \ac{IGM} and \ac{ICM}, which are highly sensitive to astrophysical feedback. With its low noise and broad frequency coverage, which is essential for separating out other signals, PICO will yield a definitive tSZ map over the full sky with a total \ac{SNR} of 1270 for the CBE and $10$\% lower for the baseline configurations (Fig.~\ref{fig:PICO_tSZ_PS}).  \comred{what is unique? Full sky? frequencies? resolution?} 
The 150,000 clusters forecast to be detected by PICO will be found in this map.
Considering the LSST gold weak-lensing sample, with a source density of 26 galaxies/arcmin${}^2$ covering 40\% of the sky, we forecast a detection of the tSZ--weak-lensing cross-correlation with \ac{SNR} = 3000.  Cross-correlations with the galaxies themselves will be measured at even higher \ac{SNR}.  At this immense significance, the signal will be broken down into dozens of tomographic redshift bins, precisely tracing the evolution of thermal pressure over cosmic time.  


\end{document}

