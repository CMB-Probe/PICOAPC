\documentclass[PICOAPC.tex]{subfiles}

\begin{document}

%To cover: Galaxy Formation, Clusters, Reionization, point sources (probably moves to a new section called 'Legacy Science')
%\subsubsection{The Formation of the First Luminous Sources} 
 
$\bullet$ {\bf The Formation of the First Luminous Sources} \hspace{0.1in} \label{sec:luminoussources}  A few hundred million years after the Big Bang, the neutral hydrogen gas permeating the Universe was reionized by photons emitted by the first luminous sources to have formed.  The nature of these sources and the exact history of this epoch are key missing links in our understanding of structure formation.  Open questions include:  were the ionizing sources primarily star-forming galaxies or more exotic sources such as supermassive black holes or annihilating dark matter? What was the mean free path of ionizing photons during this epoch?  What was the efficiency with which such photons were produced by ionizing sources?  Did the reionization epoch extend to $z \sim 15$--$20$, as has been claimed recently~\citep{Miranda2017}? With full sky coverage, multiple frequency bands, and ample sensitivity to remove foregrounds, PICO is uniquely suited to make the low-$\ell$ $EE$-spectrum measurements and reach cosmic-variance-limited precision with $\sigma(\tau)=0.002$, settling some of these questions and significantly constraining the others. Data from PICO's frequency bands above 400~GHz -- which have better than 2~arcmin resolution  -- will be used to provide clean maps for higher resolution ground-based instruments that can reconstruct the patchy $\tau$ field. No other experiment can provide these data. \\
% \comred{this needs to be improved}. 
$\bullet$ {\bf Probing the Evolution of Structures via Gravitational Lensing and Cluster Counts} \hspace{0.1in} \label{sec:gravitationallensing}   The amplitude of linear fluctuations as a function of redshift, parameterized by $\sigma_8(z)$, is a sensitive probe of physical processes affecting growth of structures in the Universe. \ac{CMB} photons are affected by, and thus probe, $\sigma_{8}(z)$ as they traverse the entire Universe. The PICO sub-percent constraints on $\sigma_8(z)$, obtained through measurements of gravitational lensing and -- independently -- using cluster counts, will translate to constraints on dark energy, models of modified gravity, baryonic feedback process, and limits on the particle content of the Universe. 

{\bf -- Gravitational Lensing} \hspace{0.1in} \label{lensing} Matter between us and the last-scattering surface deflects the path of photons through gravitational lensing, imprinting the three-dimensional matter distribution across the volume of the Universe onto the CMB maps. The specific quantity being mapped by the data is the projected gravitational potential $\phi$ that is lensing the photons. 
PICO's data will give a map of $\phi$ with \ac{SNR} $\gg1$ for each mode in the range $2 \leq L \lesssim 1000$. The \ac{SNR} on the angular power spectrum  $C_{L}^{\phi \phi}$ will be 560 and 644 for the baseline and CBE configurations, respectively; both values already account for foreground separation, and for $L \lesssim 1000$ they are the highest of any experiment in the next decade, including the higher resolution CMB-S4. 

PICO's $\phi$ map is a key ingredient in the delensing process that improves constraints on $r$~(\S~\ref{sec:inflation}) and in extracting neutrino mass constraints~(\S~\ref{sec:relics_neutrinos}). It will also be used to constrain the properties of quasars and other high-redshift astrophysical tracers of structure. 
Cross-correlations between the PICO lensing-potential map and wide-field samples of galaxies and quasars -- such as from LSST -- provide a powerful technique to measure the time dependence of the amplitude of matter fluctuations $\sigma_{8}(z)$ in tomographic redshift bins. Sub-percent accuracy is obtainable with PICO's resolution, which will give information extending to $L =1000$; see the left panel of Fig.~\ref{fig:sigma8}. 

%\begin{figure}
%\centering
%\hspace{-0.15in}
%\includegraphics[width=3in]{images/PICO_s8_lmax_PICOv4.1b_deproj0_SENS0_LSST10yrGold.pdf}
%\hspace{-0.1in}
%\includegraphics[width=2.65in,trim= 0cm -0.25cm 0cm 0cm]{images/PICOs8.png}
%\vspace{-0.14in}
%\caption{\captiontext  
%Sub-percent constraints on the evolution of $\sigma_{8}$ as a function of redshift will come from two independent PICO products: correlations between PICO's deep gravitational lensing map (Fig.~\ref{fig:lensingNoisePICO}) and LSST's gold sample of galaxies (left) and cluster counts (right). Fractional uncertainties in $\sigma_{8}$ relative to fiducial $\Lambda$CDM values are given as a function of the finest angular scale $L_{\rm max}$ of the correlation analysis for seven redshift bins (left).  The baseline and CBE configurations give essentially the same fractional errors of $\sigma_{8}(z)$ using cluster counts (right).  For LSST we assume: 10 years, 50\% sky fraction, 55 galaxies per ${\rm arcmin}^{2}$ at redshift $z<3$ with magnitude limit $i <25.3$~\citep{LSSTSciBook}, and dropout galaxies at $z>3$~\citep{dropouts} extrapolating recent Hyper Suprime-Cam observations~\cite{Schmittfull/Seljak,HSC1,HSC2}, with linear bias $b(z)=1+z$.
%\label{fig:sigma8} }
%\vspace{-0.16in}
%\end{figure}

{\bf -- Cluster Counts} \hspace{0.1in} \label{clusters}  
The distribution of galaxy clusters over redshift is one consequence of the evolution of structures and is thus a sensitive measure of $\sigma_{8}(z)$. 
%The observational quantity of interest is $dN/(dz \,\, d m) $, the number of observed clusters per redshift and per mean mass interval, from which constraints on $\sigma_{8}(z)$ can be derived. Galaxy clusters found by PICO via the \ac{tSZ} effect (\S~\ref{sec:sz}) provide a catalog with a selection function that is simple to model and thus straightforward to use for cosmological inference. PICO's catalog will provide clusters with masses above $\sim2\times10^{14} M_\odot$ out to redshifts $z\sim3$. 
We forecast that PICO will find $\sim$150,000 galaxy clusters, assuming the cosmological parameters from \planck\  and using the 70\% of sky not obscured by the Milky Way.  Information provided by the high frequency bands will mitigate the potential reduction in detection efficiency due to dust emission by cluster members~\citep{melin_2018}. 
%Redshifts will be provided by future optical and infrared surveys, while cluster masses will be inferred by optical weak lensing for clusters with $z < 1.5$ and by PICO's own CMB halo lensing data at higher redshifts (see next paragraph).  
This catalog will provide $\sigma_{8}$ with sub-percent precision for $0.5 < z < 2$ (Fig.~\ref{fig:sigma8}, right), and a neutrino mass constraint $\sigma(\sum m_{\nu}) = 14$~meV that is independent from the one coming from the CMB lensing measurements (SO3, \S~\ref{sec:relics_neutrinos}). A significant fraction of the PICO-detected clusters will also be detected by eROSITA, giving an exceptional catalog of multi-wavelength observations for detailed studies of cluster astrophysics. 


%{\bf -- Cluster Counts} \hspace{0.1in} \label{clusters}  
%The distribution of galaxy clusters over redshift is one consequence of the evolution of structures and is thus a sensitive measure of $\sigma_{8}(z)$. The observational quantity of interest is $dN/(dz \,\, d m) $, the number of observed clusters per redshift and per mean mass interval, from which constraints on $\sigma_{8}(z)$ can be derived. Galaxy clusters found by PICO via the \ac{tSZ} effect (\S~\ref{sec:sz}) provide a catalog with a selection function that is simple to model and thus straightforward to use for cosmological inference. PICO's catalog will provide clusters with masses above $\sim2\times10^{14} M_\odot$ out to redshifts $z\sim3$. We forecast that PICO will find $\sim$150,000 galaxy clusters, assuming the cosmological parameters from \planck\  and using the 70\% of sky not obscured by the Milky Way.  Information provided by the high frequency bands will mitigate the potential reduction in detection efficiency due to dust emission by cluster members~\citep{melin_2018}, an advantage of space-based observations. Redshifts will be provided by future optical and infrared surveys, while cluster masses will be inferred by optical weak lensing for clusters with $z < 1.5$ and by PICO's own CMB halo lensing data at higher redshifts (see next paragraph).  This catalog will provide $\sigma_{8}$ with sub-percent precision for $0.5 < z < 2$ (Fig.~\ref{fig:sigma8}, right), and a neutrino mass constraint $\sigma(\sum m_{\nu}) = 14$~meV that is independent from the one coming from the CMB lensing measurements (SO3, \S~\ref{sec:relics_neutrinos}). A significant fraction of the PICO-detected clusters will also be detected by eROSITA, giving an exceptional catalog of multi-wavelength observations for detailed studies of cluster astrophysics. 

%The distribution of galaxy clusters in redshift is one consequence of the evolution of structures and is thus a sensitive measure of $\sigma_{8}(z)$. The observational quantity of interest is $dN/(dz \,\, d m) $, the number of observed clusters per redshift and per mean mass, from which constraints on $\sigma_{8}(z)$ can be derived. Galaxy clusters found by PICO via the \ac{tSZ} effect (\S~\ref{sec:sz}) provide a catalog with a selection function that is simple to model and thus straightforward to use for cosmological inference. PICO's catalog will provide all clusters with masses above $\sim3\times10^{14} M_\odot$ out to redshifts $z\sim3$, as long as the clusters have started to virialize. We forecast that PICO will find $\sim$150,000 galaxy clusters, assuming the cosmological parameters from \planck\  and using the 70\% of sky not obscured by the Milky Way.  Redshifts will be provided by future optical and infrared surveys.  Cluster masses will be inferred by optical weak lensing for clusters with $z < 1.5$ and by PICO's own CMB halo lensing data at higher redshifts (see next paragraph).  This catalog will provide $\sigma_{8}$ with sub-percent precision for $0.5 < z < 2$ (Fig.~\ref{fig:sigma8}, right), and a neutrino mass constraint $\sigma(\sum m_{\nu}) = 14$~meV that is independent from the one coming from the combination of optical depth and lensing measurements (SO3, \S~\ref{sec:relics_neutrinos}). %The catalog will give the most massive clusters over the full sky.


%\comor{Now to connect to $\sigma_{8}$ quantitatively in the last sentence. Need to compare to SO. Need to explain why this is compelling. Need to explain why PICO (e.g. full sky coverage) probe?}
%\comor{Points to still hit High z sample, Numbers -- Nick and Jim should cross-check, most massive cluster all over the whole sky,
%Cosmology.}

%Calibrating the masses of clusters, i.e. determining $m(z)$, is the most uncertain step in inferring $\sigma_{8}$ and other cosmological parameters using cluster counts.  PICO will provide calibration using ``CMB halo lensing'', an approach that uses the small-scale effects of gravitational lensing due to dark matter halos around clusters and proto-clusters~\citep{2015ApJ...806..247B, 2015PhRvL.114o1302M, 2016A&A...594A..24P}. The technique is particularly effective for measuring halo masses out to high redshifts where gravitational lensing of background objects no longer works because there are no background sources. 
%The approach is illustrated in Fig.~\ref{fig:HaloLensing}, which gives the $1\sigma$ uncertainty in a halo mass measurement as a function of the object's redshift. PICO will measure the mass of individual low-mass clusters ($\sim 10^{14}$\,M$_\odot$) over a wide redshift range, and by stacking will determine the mean mass of smaller halos, with masses of $\sim 10^{13}$\,M$_\odot$, which include those hosting individual galaxies. Because the vast majority of clusters have masses that are larger than $\sim 10^{14}$\,M$_\odot$, the PICO data will provide mass calibration for all objects of interest. The flattening at high redshift reflects the fact that the technique is sensitive over a broad range of redshifts. The high-frequency PICO data, for which the resolution matches that of ground-based instruments at lower frequencies, will play an essential role in cleaning foregrounds, particularly those derived from the temperature-based estimator, which is most contaminated by foregrounds. 
%\begin{figure}[h]
%\vspace{-0.1in}
%\hspace{-0.1in}
%\parbox{3.1in}{\centerline {
%\includegraphics[width=3.0in]{images/m500lim_vs_z_pico_polar_v4.pdf} } }
%\hspace{0.in}
%\parbox{3.4in}{
%\caption{\captiontext 
%PICO will provide mass calibration for individual clusters and proto-clusters with mass as low as $10^{14}M_\odot$ at $z>2$ using `halo lensing'. Curves for different CMB signal correlations (red) give the $1\sigma$ sensitivity of an optimal mass filter~\citep{2015A&A...578A..21M} as a function of $z$.  The curves are flat at high redshift, demonstrating that the technique probes a broad range of redshifts. For PICO, the $EB$ and $TT$ estimators are equivalent, offering important cross-validation of measurements because the systematics are very different for temperature and polarization. 
%\label{fig:HaloLensing} } }
%\vspace{-0.2in}
%\end{figure}

%Beyond its role in calibrating masses for cluster counts, PICO's halo lensing measurements will also be a unique tool for measuring the relation between galaxies and their dark matter halos during the key epoch of cosmic star formation at $z\geq 2$, which is not reachable by other means.  This will provide valuable insight into the role of environment on galaxy formation during the rise to and fall from the peak of cosmic star formation at $z\sim 2$. 

%the mass sensitivity of PICO using a spatial filter optimized for extracting the mass of halos \citep{2015A&A...578A..21M}.  The curves give the $1\sigma$ noise in a mass measurement through the filter as a function of redshift. 


%\subsubsection{Constraining Feedback Processes through the Sunyaev--Zeldovich Effect}
%Compton-$y$ map and tSZ auto-power spectrum} 
%\label{sec:sz}
%\label{ymap}  


%\subsubsection{Constraining Structure Growth and Galaxy Formation via the Sunyaev-Zeldovich (SZ) Effects} 
% \label{sec:sz}
$\bullet$ {\bf Constraining Feedback Processes through the Sunyaev--Zeldovich Effect} \hspace{0.1in} \label{sec:sz}
About 6\% of CMB photons are Thomson-scattered by free electrons in the \ac{IGM} and \ac{ICM}, and a fraction of these are responsible for the thermal and kinetic Sunyaev--Zeldovich effects (tSZ and kSZ)~\citep{zeldovich69,SZ1972}. The amplitude of the tSZ is proportional to the integrated electron pressure along the line of sight, and it thus contains information about the thermodynamic properties of the \ac{IGM} and \ac{ICM}, which are highly sensitive to astrophysical feedback. With its low noise and broad frequency coverage, which is essential for separating out other signals, PICO will yield a definitive tSZ map over the full sky with a total \ac{SNR} of 1270 for the CBE and $10$\% lower for the baseline configurations (Fig.~\ref{fig:PICO_tSZ_PS}).  \comred{what is unique? Full sky? frequencies? resolution?} 
The 150,000 clusters forecast to be detected by PICO will be found in this map.
Considering the LSST gold weak-lensing sample, with a source density of 26 galaxies/arcmin${}^2$ covering 40\% of the sky, we forecast a detection of the tSZ--weak-lensing cross-correlation with \ac{SNR} = 3000.  Cross-correlations with the galaxies themselves will be measured at even higher \ac{SNR}.  At this immense significance, the signal will be broken down into dozens of tomographic redshift bins, precisely tracing the evolution of thermal pressure over cosmic time.  

%This is nearly two orders of magnitude higher \ac{SNR} than \planck , which already gave data with much higher \ac{SNR} than ground-based experiments. The tens of thousands of clusters forecast to be detected by PICO will be found in this $y$ map (\S~\ref{sec:gravitationallensing}).

%%%%%%%%%%%%%%%%%%%%
%\subsubsection{Constraining Feedback Processes: Compton-$y$ map and tSZ auto-power spectrum} 
%\label{ymap}  

%\begin{figure}[h]
%\hspace{-0.1in}
%\parbox{3.1in}{\centerline{
%\includegraphics[width=3.0in]{images/PICO_tSZ_PS_plot.pdf} } }
%\hspace{0.in}
%\parbox{3.4in}{
%\caption{\captiontext  
%The PICO $y$-map will give a tSZ power spectrum with an \ac{SNR} of 1270 (green, $1\sigma$ per $\ell$ mode), which is nearly 100 times larger than from \planck\ (blue). Binning the data (not shown), as was done for \planck\, would further increase the \ac{SNR}.  We also show current measurements by the ground-based SPT and ACT~\citep{Sievers2013,George2015}. In these forecasts we reconstruct the Compton-$y$ field from maps that include Galactic foregrounds, CMB fluctuations, and PICO CBE noise using the needlet internal linear combination algorithm~\citep{Delabrouille2009}. The input maps use the \planck~sky model~\cite{delabrouille/etal:2013}.
%%The black curve shows the simulated tSZ power spectrum signal.  The light green shaded region shows the error bars for PICO at each multipole, i.e., with no binning, as determined from NILC analysis of full-sky simulations.  The blue points show the current constraints from Planck, which have been averaged into broad multipole bins.  The orange and dark green points show the constraints from ACT and SPT, respectively, at a single multipole of $\ell=3000$.  The overall PICO $S/N = 1270$, nearly two orders of magnitude larger than current measurements.
%\label{fig:PICO_tSZ_PS} } }
%\vspace{-0.2in}
%\end{figure}

%Strong constraints on models of astrophysical feedback will be obtained from the analysis of the PICO $y$-map, both from its auto-power spectrum and from cross-correlations with galaxy, group, cluster, and quasar samples. 
% Like the CMB-lensing map described above, the legacy value of the PICO $y$-map will be immense.  
%As an example, we forecast the detection of cross-correlations between the PICO $y$-map and galaxy weak-lensing maps constructed from LSST and WFIRST data.  Considering the LSST gold weak-lensing sample, with a source density of 26 galaxies/arcmin${}^2$ covering 40\% of the sky, we forecast a detection of the tSZ--weak-lensing cross-correlation with \ac{SNR} = 3000.  Cross-correlations with the galaxies themselves will be measured at even higher \ac{SNR}.  At this immense significance, the signal can be broken down into dozens of tomographic redshift bins, precisely tracing the evolution of thermal pressure over cosmic time.  For PICO and WFIRST (assuming 45 galaxies/arcmin${}^2$ covering 5.3\% of the sky), we forecast \ac{SNR} = 1100 for the tSZ-weak lensing cross-correlation.  The WFIRST galaxy sample extends to higher redshift, and thus this high-\ac{SNR} measurement will allow the evolution of the thermal gas pressure to be probed to $z \approx 2$ (the peak of the cosmic star formation history) and beyond.  These measurements will revolutionize our understanding of galaxy formation and evolution by distinguishing between models of feedback energy injection at high significance.  Additional cross-correlations of the PICO $y$-map with quasar samples, filament catalogs, and other large-scale structure tracers will provide valuable information on baryonic physics that is complementary to inferences from the lensing cross-correlations described earlier.  
%Finally, beyond Compton-$y$, PICO's CMB halo lensing measurements will also be a unique tool for measuring the relation between galaxies and their dark matter halos during the key epochs of cosmic star formation at $z\geq 2$, not reachable by other means.  This will provide valuable insight into the role of environment on galaxy formation during the rise to and fall from the peak of cosmic star formation at $z\sim 2$. %\comor{is this already covered earlier in Marcel's lensing cross-correlations text?}



\end{document}


%Measurements of the CMB reveal structure imprinted not only at the early time of recombination, but also at nearly every significant ensuing epoch in cosmic history.  In particular the matter between us and the CMB last-scattering surface will deflect the path of CMB photons, a process known as gravitational lensing.  Although the lensing of the CMB is a weak signal, targeted statistical estimators enable its extraction.  Measurements of the lensing signal have rapidly progressed, from the first detections in 2007-8 \citep{2007PhRvD..76d3510S, 2008PhRvD..78d3520H} to the recent $40\sigma$ measurement by the {\it Planck} team \cite{2018arXiv180706210P}.  When applied to a rich dataset such as that expected from the PICO satellite, these estimators will provide a map of all the matter in the Universe in projection, with the most sensitivity at redshift $z \simeq 2$ and down to scales of approximately ten arcminutes.  

%Forecasts show that the power spectrum of lensing in the PICO CMB map can be detected at approximately 580$\sigma$ or 650$\sigma$ for the requirement or CBE configurations, respectively.  Such high-S/N measurements are more than an order of magnitude improvement over the current state of the art, obtained by the {\it Planck} team.  The legacy value of the PICO CMB lensing map is immense, as has already been seen with {\it Planck}, particularly through the high-S/N cross-correlation science that will be enabled.  For example, tomographic cross-correlations of the PICO CMB lensing map with samples of galaxies and quasars will yield constraints on structure formation out to redshifts inaccessible to galaxy surveys or galaxy weak lensing maps on their own.  These measurements will yield constraints on dark energy, modified gravity, and neutrino mass (complementary to the neutrino mass inferred from the CMB lensing auto-power spectrum described earlier).  In addition, PICO CMB lensing cross-correlations will yield constraints on the properties of quasars and other high-redshift astrophysics, e.g., a precise determination of the quasar bias (and hence host halo mass) as a function of their properties, such as (non-)obscuration.  We provide further quantitative cross-correlation forecasts below.  Fig.~\ref{fig:lensingNoisePICO} shows per-mode noise curves for the reconstruction of CMB lensing from PICO, demonstrating the wide range of angular scales over which the matter density field will be mapped.  \textbf{Add brief discussion of foreground robustness demonstrated in Fig.~\ref{fig:lensingNoisePICO}}

%\begin{figure}[!htb]
%\centering
%\includegraphics[width=4cm]{images/example}
%\caption{example}
%\label{fig:im_3}
%\end{figure}
