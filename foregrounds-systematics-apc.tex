\documentclass[PICOAPC.tex]{subfiles}

%\newcolumntype{L}[1]{>{\raggedright\let\newline\\\arraybackslash\hspace{0pt}}m{#1}}
%\newcolumntype{K}[1]{>{\raggedright\centering\arraybackslash}m{#1}}

\begin{document}

Controlling foregrounds and systematic effects are key for the success of any experimental endeavor striving to achieve $\sigma(r) \lesssim 1 \times 10^{-3}$. \\
$\bullet$ \hspace{0.1in}  PICO has the highest sensitivity of any next-decade CMB experiment, and the most frequency bands compared to any imaging instrument. It is thus more suitably equipped to handle foreground complexities. Higher sensitivity will translate to higher \ac{SNR} in detecting systematic effects. \\ 
$\bullet$ \hspace{0.1in}  \citet{pico_report} have shown that frequencies above 400~GHz may be essential for removing large angular scale foregrounds (see also \citet{hensley_2017}). They have also shown that for several realistic sky models PICO should be able to satisfy its $r$ detection requirement.  \\
%Based on community experience with both hardware and analysis of data we make the following points.  \\
$\bullet$ \hspace{0.1in}  Relative to other platforms, a space-based mission provides the most thermally stable platform,  a prerequisite for improved control of systematic effects. PICO's orbit at L2 is among the most thermally stable of possible orbits. \\
$\bullet$ \hspace{0.1in} PICO's sky scan pattern gives strong data redundancy, which enables numerous cross-checks. Each of the 12,996 detectors makes independent maps of the $I,\,Q$, and $U$ Stokes parameters enabling many comparisons within and across frequency bands, within and across sections of the focal plane, and within and across bolometers that have either the same or different polarization sensitivities. Half the sky is scanned every two weeks, and the entire sky is scanned in 6 months. \\
$\bullet$ \hspace{0.1in}  The scan pattern gives almost continuous scans of planets and large amplitude ($\geq 4$~mK) CMB dipole signals~\citep{picoweb_dipole}. These features result in continuous, high \ac{SNR} calibration and antenna-pattern characterization. \\
We direct the reader to the mission study report for more details on our work on foreground rejection and characterization of systematic effects for PICO~\citep{pico_report}.

\end{document}


