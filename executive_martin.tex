\documentclass[PICOReport.tex]{subfiles}

\begin{document}

%% Beginning of Martin

The \ac{CMB} comes to us from the furthest reaches of the observable Universe, and its photons experience all of cosmic history.  Created when the Universe was a hotter, simpler place, CMB photons probe fundamental physics, provide exquisite measurements of the constituents of the cosmos, and tests of relativity.  On their journey they feel the impact of the gravitational potentials formed from the assembling cosmic web of superclusters, clusters and galaxies.  They interact with the ionized gas in the inter- and circum-galactic medium, gas that eventually fuels star and galaxy formation.  Superposed upon the CMB is the emission from multiple extragalactic sources and from our Galaxy.  All of this leaves an imprint which sensitive measurements can disentangle so that CMB studies impact essentially every aspect of cosmology and astrophysics.

Building upon a long legacy of successful measurements, the next decade holds tremendous potential for new, exciting \ac{CMB} discoveries.  Such discoveries, delivered by the Probe of Inflation and Cosmic Origins (PICO), promise to be revolutionary, affecting physics, astrophysics, and cosmology. PICO is an imaging polarimeter that will scan the sky for 5 years with 21 frequency bands spread between 21 and 800~GHz; see Table~\ref{tab:specs}. It will produce 10 independent full sky surveys of intensity and polarization with a final combined-map noise level equivalent to 3250 \planck\ missions for the baseline required specifications, though in our current best-estimate (CBE) it would perform as 6400 \planck\ missions.  It will produce the first ever full sky polarization maps at frequencies above 350 GHz, and it will have diffraction limited resolution of 1' at 800~GHz. 

With these unprecedented capabilities, unmatched by any other existing or proposed platform, PICO will have compelling and broad science deliverables. The mission will respond to 7 science objectives (SOs), which are listed in Table~\ref{tab:STM}. Delivering this set was the basis for selecting PICO's design and for setting instrument requirements. But PICO's science reach is far broader than the baseline set. 

%\comor{add section references here, or a table of contents?}
PICO could determine the energy scale of inflation and give a first, direct probe of quantum gravity; this is SO1 (\S~\ref{sec:fundamentalsci). If the signal is not detected PICO will constrain broad classes of inflationary models, and exclude at $10\sigma$ models for which the characteristic scale in the potential is the Planck scale (SO1 and SO2). The combination of PICO with LSST can constrain features in the inflationary potential, the field content during inflation and could rule out all models of slow-roll single-field inflation, marking a watershed in studies of inflation. 

The mission will have a deep impact on particle physics by measuring the expected sum of the neutrino masses in two independent ways, each with at least $4\sigma$ confidence, rising to $7\sigma$ if the sum is near 0.1~eV (SO3). \comor{ this is not quite right; the cluster constraint will require follow up, so the second way is not from the mission} The measurements will either detect or strongly constrain deviations from the standard model of particle physics by counting the number of light particles in the early universe at an energy range that is up to 400 times higher than available today \comor{this needs to be modified} (SO4). The data will constrain dark matter candidates by pushing down \planck\ constraints on the dark matter annihilation cross section by a factor of 25, specifically at low energy scales that are not accessible to direct detection experiments. The data will probe the existence of cosmic fields that could give rise to cosmic birefringence. \comor{what about dark energy?}
%The data will probe extensions of the standard model, including the existence of cosmic fields that could give rise to cosmic birefringence, and of primordial magnetic fields. \comor{what about dark energy?}

PICO will transform our knowledge of the structure and evolution of the universe. It will measure the redshift at which the universe reionized, impacting physical models describing when and how the first luminous objects formed (SO5). \comor{more quantitative regarding $\tau$?} It will make a map of the projected matter throughout the universe with a signal-to-noise ratio exceeding 500. This will constrain the mass of dark matter halos hosting galaxies, groups, and clusters from the present day to the very first such objects. \comor{more quantitative?} The map will be cross-correlated with other next-decade galaxy surveys, such as LSST, to measure the growth of large-scale structure with sub-percent precision.  An extraordinary amount of information about the role of 'energetic feedback' on structure formation will come from correlating PICO's map of the thermal Sunyaev-Zel?dovich effect with galaxy and lensing maps from WFIRST and LSST. The correlation -- forecast to have a signal-to-noise of 3000 with LSST weak lensing -- will enable measurements in dozens of tomographic redshift bins, giving extraordinarily detailed information about the evolution of thermal energy injection over cosmic time.

Magnetic fields thread galaxies and affect their structure and evolution, but the origins of these magnetic fields is a hotly debated question. PICO will test whether galactic magnetic fields have been seeded by primordial magnetic fields of cosmic origin. It will map the entire Milky Way in polarization with unprecedented detail at many frequency bands. Such maps are not planned by any other survey, and can not be produced other than in space.  From these unique maps we will map the Galactic magnetic field structure elucidating the relative roles of turbulence and magnetic fields in the observed low star formation efficiency, and we will strongly constrain the properties of the diffuse interstellar medium, including dust grain composition, temperature and emissivities (SO6 and SO7).

The cosmic infrared background (CIB) encodes information from star-formation, obscured or unobscured, across cosmic time and PICO will improve upon existing measurements of CIB clustering by an order of magnitude \comor{what is CIB clustering}.  By discovering 150,000 clusters, 50,000 proto-clusters (up to z=4.5), and 4500 strongly lensed galaxies (up to z=5), PICO will enable a unique view into early galaxy and cluster evolution. The window PICO provides because of its high frequency bands is entirely unique and not available to any other experiment.  PICO's sub-mm maps will provide ancillary information (including polarization) for all future measurements anywhere on the sky.  \comor{dark energy science removed; put back?}
%By discovering 150,000 clusters distributed over cosmic time PICO data, together with future cluster redshift surveys will constrain the dark energy equation of state with constraints similar to other next decade surveys,

This scientifically ground-breaking mission is based on technologies that are being used actively today by ground- and balloon-based experiments, but over a more restricted range of frequency band. These technologies will continue to mature by a host of recently funded sub-orbital activities well before the mission's Phase-A. Section~\ref{sec:??}
\comor{now here}
\comor{include: single instrument, simple scan; need funding for foregrounds and systematics}

All the implementation aspects are mature, benefiting from thousands of person-years of experience studying the sky at these wavelengths. These span over more than 50 years of mapping the CMB and include three enormously successful space missions. This combined experience unambiguously shows that the unlimited frequency coverage and thermally benign environment aboard a space-based platform give unparalleled capability to separate the combination of galactic and cosmological signals and to control systematic uncertainties. These qualities, which are critical ingredients for any next-decade experiment, make PICO the optimal platform for a next generation CMB experiment.

%% End of Martin


%\comor{broad science, unique mission, nothing better in the foreseable future, complementing and enriching other science in the next decade, comparatively cheap, within cost, using existing technologies, relying on extensive community experience both on the ground and in space}

% PICO's data will enrich and complement other astrophysical surveys in the next decade.

%We note that if there {\it is} a detection of the \ac{IGW} signal with $r=0.001$, PICO will make it with high significance in multiple independent patches of the sky. 


%A detection would strongly benefit from confirmation at {\it both} angular scales -- a goal that is beyond the capabilities of ground-based instruments -- {\it and}, for the $\ell = 80$ peak, in several independent patches of the sky -- a goal that is currently not planned for any next decade instrument. 

\end{document}

%see Fig.~\ref{fig:im_1}

%\begin{figure}[!htb]
%\centering
%\includegraphics[width=4cm]{images/example}
%\caption{example}
%\label{fig:im_1}
%\end{figure}
