\documentclass[PICOAPC.tex]{subfiles}
 
\begin{document}

The set of physical parameters and observables that derive from the PICO \ac{SOs} place 
requirements on the depth of the mission, the fraction of sky the instrument scans, the frequency range the 
instrument probes and the number of frequency bands, the angular resolution provided by the reflectors, and
the specific pattern with which PICO will observe the sky.  \\
%combination of PICO's diverse science goals are achievable using a single instrument that executes one, continuous, simple observing pattern of the entire sky. This pattern integrates noise down to unprecedented levels, and provides for multiple checks of possible systematic errors in the data analysis. \\
%
$\bullet$ {\bf Depth} \hspace{0.1in} We quantify survey depth in terms of the RMS fluctuations that would give
an \ac{SNR} ratio of~1 in 1\arcmin$\times$1\arcmin\ sky pixel.  The science objective driving 
the depth requirement is SO1, the search for the inflationary signal, which 
requires a combined depth of 0.87~\microkamin. This requirement is a combination of the low level of the signal, the need
to separate the various signals detected in each band, and the need to detect and subtract systematic effects 
to anticipated levels.  The map depth requirement flows to instrument sensitivity requirements (Table~\ref{tab:STM}) and to the mission duration requirement (5 years), assuming $95\%$ survey efficiency. \\
% The \ac{CBE} value is 0.61~\microkamin\ coming from a realistic estimate of detector noise, and giving 40\% margin on mission performance. \\
%The required depth assumes full sky coverage, which is d. We expect to only use 50-60\% of the full sky, which do not include the brightest emission from the Milky Way, for the data analysis leading to a constraint on $r$. But as discussed below, other science objectives, require a scan of the Galaxy, leading to the requirement  
%To achieve this combined depth we implement the focal plane of detectors listed inTable~\ref{tab:reqs} The requirement on performance includes the following factors  a factor of 1.5 degradation in noise relative to our baseline expectations. 
%
$\bullet$ {\bf Sky Coverage} \hspace{0.1in} There are several \ac{SOs} driving a full-sky survey for PICO. The term `full-sky' refers to the entire area of sky available after separating astrophysical sources of confusion. In practice this implies an area of 50--70\% of the sky for probing non-Galactic signals, and 100\% of sky for achieving the Galactic science goals. 

(1) Probing the optical depth to the epoch of reionization (SO5) requires full sky coverage as the signal peaks in the $EE$ power spectrum on angular scales of 20$^\circ$ to 90$^\circ$ $(2 \leq \ell \leq 10)$. Measuring this optical depth to limits imposed by cosmic variance\cref{CVL} is key for minimizing the error on the neutrino-mass measurement. 
%\comblue{and for other astrophysical surveys} 

(2) The inflationary $BB$ power spectrum (SO1) has local maxima in the `reionization peak' ($ 2 \leq \ell \leq 10$), and in the `recombination peak' ($ \ell \simeq 80$)~(Fig.~\ref{fig:clbb}). A detection would strongly benefit from confirmation at {\it both} angular scales. Measurements of the reionization peak are currently beyond the capabilities of ground-based instruments. A detection would also strongly benefit from confirmation  {\it in several independent patches of the sky}. This is achievable with PICO through observing the recombination peak in several small (3--5\% sky fraction) patches of the sky. No similar capability is currently planned for any next-decade instrument.  

%(3) The PICO constraint on $N_{\rm eff}$ (SO4) requires all the $\ell$ modes available in the $TT,\, TE$, and $EE$ power spectra, limited by cosmic variance at $\ell=3500\,(2500)$ for $TT\,(EE)$. To achieve this, full sky coverage is required.  
(3) The PICO constraint on $N_{\rm eff}$ (SO4) requires determination of the $TT,\, TE$, and $EE$ power spectra, limited by cosmic variance at $\ell=3500\,(2500)$ for $TT\,(EE)$. To achieve this, full sky coverage is required.  

(4) Achieving the targeted neutrino mass limits (SO3), giving two independent $4\sigma$ constraints on the minimal sum of 58~meV, requires a lensing map and cluster counts from as large a sky fraction as possible. 

(5) PICO's survey of the Galactic plane and regions outside of it is essential to achieving its Galactic structure  and star-formation science goals (SO6, 7). \\
%
$\bullet$ {\bf Frequency Bands} \hspace{0.1in} The multitude of astrophysical signals that PICO will characterize determine the frequency range and number of bands that the mission requires. The Galactic and cosmological signals are separable using their spectral signatures. The cosmological signals peak in the frequency range between 60 and 300~GHz. Galactic signals, specifically the make-up of Galactic dust (SO6), require spectral characterization at frequencies between 100 and 800~GHz. Simulations indicate that 21 bands, each with 25\% bandwidth, that are spread across the range 20--800~GHz can achieve the separation between Galactic and cosmological signals at the level of fidelity required by PICO~(\S~\ref{sec:signal_separation}). \\
%
$\bullet$ {\bf Resolution} \hspace{0.1in} 
Several \ac{SOs} require the resolution per frequency listed in Table~\ref{tab:specs}. To reach $\sigma(r) = 1\times10^{-4}$ we will need to `delens' the $B$-mode map, as described in \S~\ref{sec:fundamentalsci} and~\S~\ref{sec:gravitationallensing}. Delensing efficacy is a function of noise and resolution. For PICO, the combination of the two gives between 73 and 85\% delensing, which is adequate for achieving our \ac{SOs}. The process of delensing may be affected by contamination from Galactic dust. It is thus required to map Galactic dust to at least the same resolution as in the main CMB bands.  Higher resolution is mandated by SO6 and 7, which require resolution of 1\arcmin\ at 800~GHz. 

The constraints on the number of light relics (SO4) will be extracted from the $TT,\,TE$ and $EE$ power spectra at $\ell \lesssim 4000$, which requires the resolution specified in Table~\ref{tab:specs}. \\
%We have thus chosen to implement diffracted-limited resolution between 20 and 800~GHz. \\
%
$\bullet$ {\bf Sky Scan Pattern} \hspace{0.1in} 
Control of polarization systematics uncertainties at anticipated levels is enabled by: (1) making $I$, $Q$, and $U$ Stokes-parameter maps of the entire sky from each independent detector; (2) by enabling sub-percent absolute gain calibration of the detectors through observations of the CMB dipole; and (3) by enabling cross-checks on the results through comparing multiple cuts of the data, a process known as `jack-knife test'.   With these requirements we chose a sky scan pattern (\S~\ref{sec:survey_design}) that enables each detector to scan a given pixel of the sky in a multitude of directions, satisfying requirement (1). The scan gives large amplitude CMB dipole signals in spacecraft rotations throughout the lifetime of the mission, satisfying requirement (2). With PICO's sky scan pattern, more than 50\% of the sky is  scanned within two weeks of the start of the survey. The entire sky is surveyed within 6 months, and then this pattern repeats. Thus the PICO scan pattern gives 10 independent maps and multiple ways to perform data jack-knives, satisfying requirement (3).   
% need to be more quantitative. 

 
\end{document}

%\begin{figure}[!htb]
%\centering
%\includegraphics[width=4cm]{images/example}
%\caption{example}
%\label{fig:im_3}
%\end{figure} 

