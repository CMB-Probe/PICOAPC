\documentclass[PICOAPC.tex]{subfiles}

\begin{document}
 
$\bullet$ {\bf The Formation of the First Luminous Sources} \hspace{0.1in} \label{sec:luminoussources}  Measurements of the optical depth to reionization $\tau$ will illuminate the nature of the first luminous sources and the exact history of the reionization epoch, both of which are key missing links in our understanding of structure formation~\citep{alvarez_swp}.  With full sky coverage, multiple frequency bands, and ample sensitivity to remove foregrounds, PICO is uniquely suited to reach cosmic-variance-limited precision with $\sigma(\tau)=0.002$. Data from PICO's frequency bands above 400~GHz -- which have better than 2~arcmin resolution  -- will be used to provide clean maps for higher resolution ground-based instruments that can reconstruct the patchy $\tau$ field. No other experiment can provide these data. \\
%
$\bullet$ {\bf Probing the Evolution of Structures via Gravitational Lensing and Cluster Counts} \hspace{0.1in} \label{sec:gravitationallensing}   
%The amplitude of linear fluctuations as a function of redshift, parameterized by $\sigma_8(z)$, is a sensitive probe of physical processes affecting growth of structures in the Universe. 
PICO will give sub-percent constraints on $\sigma_8(z)$, the amplitude of linear fluctuations as a function of redshift, through measurements of gravitational lensing of the CMB photons and independently by using cluster counts. 
PICO will have an \ac{SNR} of more than 560 for measurement of $C_{L}^{\phi \phi}$, the angular power spectrum of the projected gravitational potential $\phi$ that is lensing the photons. This is the highest of any foreseeable CMB experiment in the range $2 \leq L \lesssim 1500$. When combined with LSST data the measurement will give $\sigma_8(z) <0.5\%$ in each of six $z$ bins for $z>0.5$~\cite{pico_report}.
The mission will find $\sim$150,000 galaxy clusters, and this catalog will provide $\sigma_{8}(z) < 1\%$ for each of eight bins in $0.5 < z < 2$, and a neutrino mass constraint $\sigma(\sum m_{\nu}) = 14$~meV that is independent from the one coming from $C_{L}^{\phi \phi}$. A significant fraction of the PICO-detected clusters will also be detected by eROSITA, giving an exceptional catalog of multi-wavelength observations for detailed studies of cluster astrophysics. The constraints on $\sigma_{8}$ will translate to constraints on dark energy, modified gravity, baryonic feedback process, and limits on the particle content of the Universe. \\
%
$\bullet$ {\bf Constraining Feedback Processes through the Sunyaev--Zeldovich Effect} \hspace{0.1in} \label{sec:sz}
The thermal SZ (tSZ) effect probes the integrated electron pressure along the line-of-sight.  PICO will detect 150,000 clusters through their tSZ signature, the largest catalog of any proposed CMB experiment, including thousands of high-redshift objects that are undetectable via X-ray emission.  PICO will also provide the only full-sky, high-\ac{SNR} tSZ map of any proposed CMB experiment.  The cross-correlation of this map with the LSST gold weak-lensing sample (26 gal/arcmin$^2$ over 40\% of the sky) will be detected at \ac{SNR}=3000, yielding a precise tomographic reconstruction of the evolution of thermal pressure over cosmic time.
%$\bullet$ {\bf Constraining Feedback Processes through the Sunyaev--Zeldovich Effect} \hspace{0.1in} \label{sec:sz}
%PICO will detect 150,000 clusters through the thermal SZ effect (tSZ) which measures integrated electron pressure along the line of sight.  Cross-correlation of the tSZ map with the LSST gold weak-lensing sample (26 gal/arcmin$^2$ over 40\% of the sky) should yield a signal with and \ac{SNR} of 3000.  The signal will be broken down into dozens of tomographic redshift bins, precisely tracing the evolution of thermal pressure over cosmic time.

%About 6\% of CMB photons are Thomson-scattered by free electrons in the \ac{IGM} and \ac{ICM}, and a fraction of these are responsible for the thermal and kinetic Sunyaev--Zeldovich effects (tSZ and kSZ)~\citep{zeldovich69,SZ1972}. The amplitude of the tSZ is proportional to the integrated electron pressure along the line of sight, and it thus contains information about the thermodynamic properties of the \ac{IGM} and \ac{ICM}, which are highly sensitive to astrophysical feedback. With its low noise and broad frequency coverage, which is essential for separating out other signals, PICO will yield a definitive tSZ map over the full sky with a total \ac{SNR} of 1270 for the CBE and $10$\% lower for the baseline configurations (Fig.~\ref{fig:PICO_tSZ_PS}).  \comred{what is unique? Full sky? frequencies? resolution?} 
%The 150,000 clusters forecast to be detected by PICO will be found in this map.
%Considering the LSST gold weak-lensing sample, with a source density of 26 galaxies/arcmin${}^2$ covering 40\% of the sky, we forecast a detection of the tSZ--weak-lensing cross-correlation with \ac{SNR} = 3000.  Cross-correlations with the galaxies themselves will be measured at even higher \ac{SNR}.  At this immense significance, the signal will be broken down into dozens of tomographic redshift bins, precisely tracing the evolution of thermal pressure over cosmic time.  


\end{document}

