%
% sensitivity table
% This is two tables in minipage environment
%
\begin{table}[tb]
\hskip3cm
\begin{minipage}[t]{0.3\textwidth}
\caption{\textbf{Mission Parameters}\label{tab:specs}}
\begingroup
%\openup 5pt
\newdimen\tblskip \tblskip=5pt
\nointerlineskip
\vskip -5mm
\footnotesize %\footnotesize
\setbox\tablebox=\vbox{
    \newdimen\digitwidth
    \setbox0=\hbox{\rm 0}
    \digitwidth=\wd0
    \catcode`*=\active
    \def*{\kern\digitwidth}
%
    \newdimen\signwidth
    \setbox0=\hbox{+}
    \signwidth=\wd0
    \catcode`!=\active
    \def!{\kern\signwidth}
%
\halign{%
\hbox to 1.8in{#\leaderfil}\tabskip=0.6em plus 0.6em&
#\hfil\tabskip=0pt\cr
\noalign{\doubleline}
\multispan2Combined polarization map depth (rms noise in $1\times1$ arcmin$^{2}$ pixel):\hfil\cr
\quad Baseline&0.87 $\mu$K$_{\rm CMB}$ arcmin equivalent to 3300 \textit{Planck} missions\cr
\quad CBE$^{a}$&0.61 $\mu$K$_{\rm CMB}$ arcmin equivalent to 6400 \textit{Planck} missions\cr
Survey duration / start & 5\,yrs / 2029 \cr
Orbit type & Sun-Earth L2 \cr
Launch mass & 2147\,kg \cr
Total power &1320\,W \cr
Data rate & 6.1\,Tbits/day \cr
Cost&\$\,958M\cr
%Launch&2029\cr
\noalign{\vskip 5pt\hrule\vskip 3pt}
\noalign{$^{a}$ CBE = Current best estimate.} % footnote to table
} % close halign
} % close vbox
\endPlancktable
\endgroup
%\end{table}
\end{minipage}
\hskip 3cm
\begin{minipage}[t]{0.5\textwidth}
\caption{\textbf{Frequency Bands, Resolution, and Noise Level}\label{tab:spec_bands}}
\begingroup
%\openup 5pt
\newdimen\tblskip \tblskip=5pt
\nointerlineskip
\vskip -5mm
\footnotesize %\footnotesize
\setbox\tablebox=\vbox{
    \newdimen\digitwidth
    \setbox0=\hbox{\rm 0}
    \digitwidth=\wd0
    \catcode`*=\active
    \def*{\kern\digitwidth}
%
    \newdimen\signwidth
    \setbox0=\hbox{+}
    \signwidth=\wd0
    \catcode`!=\active
    \def!{\kern\signwidth}
%
\halign{
\hbox to 1.8in{#\leaderfil}\tabskip=0.6em plus 0.6em&
\hfil#\hfil&
\hfil#\hfil&
\hfil#\hfil&
\hfil#\hfil&
\hfil#\hfil&
\hfil#\hfil&
\hfil#\hfil&
\hfil#\hfil&
\hfil#\hfil&
\hfil#\hfil&
\hfil#\hfil&
\hfil#\hfil&
\hfil#\hfil&
\hfil#\hfil&
\hfil#\hfil&
\hfil#\hfil&
\hfil#\hfil&
\hfil#\hfil&
\hfil#\hfil&
\hfil#\hfil&
\hfil#\hfil\tabskip=0pt\cr
\noalign{\doubleline}
Frequency [GHz]&21&25&30&36&43&52&62&75&90&108&129&155&186&223&268&321&385&462&555&666&799\cr
FWHM [arcmin]&38.4&32.0&28.3&23.6&22.2&18.4&12.8&10.7&9.5&7.9&7.4&6.2&4.3&3.6&3.2&2.6&2.5&2.1&1.5&1.3&1.1\cr
\noalign{Polarization map depth:}
\quad Baseline  [$\mu$K$_{\rm CMB}$\,arcmin]&23.9&18.4&12.4&*7.9&*7.9&*5.7&*5.4&*4.2&*2.8&*2.3&*2.1&*1.8&*4.0&*4.5&*3.1&*4.2&4.5&9.1&45.8&*177&1050\cr
\quad CBE$^a$  [$\mu$K$_{\rm CMB}$\,arcmin]&16.9&13.0&*8.7&*5.6&*5.6&*4.0&*3.8&*3.0&*2.0&*1.6&*1.5&*1.3&*2.8&*3.2&*2.2&*3.0&3.2&6.4&32.4&*125&*740\cr
\quad Baseline  [ Jy/sr]&*8.3&10.9&11.8&12.9&19.5&23.8&45.4&58.3&59.3&77.3&96.0&119&433&604&433&578&429&551&1580&2080&2880\cr
\quad CBE$^a$  [Jy/sr]&*5.9&*7.7&*8.3&*9.2&13.8&16.8&32.1&41.3&41.8&53.5&69.3&*84&302&436&304&411&303&387&1120&1470&2040\cr
\noalign{\vskip 5pt\hrule\vskip 3pt}
} % close halign
} % close vbox
\endPlancktable
\endgroup
%\end{table}
\end{minipage}
\end{table}

\vspace{8mm}
%
%
%   -----------------------------------------------------------------------------------------------------------------
%
\begin{table}
\caption{\textbf{Science Traceability Matrix (STM) }}\label{tab:STM}
%\small
\footnotesize
\begin{tabular}{@{}lcccccccc@{}}
%\noalign{\vskip 2mm}
\hline
\noalign{\vskip 2mm}    
% Header line 1
%\rowcolor[HTML]{EFEFEF} 
\multicolumn{1}{c}{\multirow{2}{1in}{\centering \bf Science Goals (from NASA Science Plan)}}&
\multicolumn{1}{c}{\multirow{2}{2in}{\centering \bf Science Objectives}}& 
\multicolumn{3}{c}{\bf Scientific Measurement Requirements}&
\multicolumn{1}{c}{}&
\multicolumn{2}{c}{\bf Instrument (single instrument, single mode)}&
\multicolumn{1}{c}{\multirow{2}{1.75in}{\centering \bf Mission Functional Requirements}} 
\\
%Header line 2
\noalign{\vskip 2mm}    
\cline{3-5}\cline{7-8}
\noalign{\vskip 2mm}    
%\rowcolor[HTML]{EFEFEF} 
\multicolumn{1}{c}{} &
\multicolumn{1}{c}{} &
\multicolumn{1}{c}{Model Parameters} &
\multicolumn{1}{c}{Physical Parameters} & 
\multicolumn{1}{c}{Observables} &
\multicolumn{1}{c}{} &
\multicolumn{1}{c}{Functional Requirements} &
\multicolumn{1}{c}{Projected Performance} & 
\\
% Line SO1
\noalign{\vskip 2mm}    
\hline
\multicolumn{1}{l}{\multirow{2}{1in}{\vskip5pt \textbf{\textit{Explore how the Universe began: Inflation}}}}&
\multicolumn{1}{l}{\parbox[t]{2in}{\textbf{SO1}. Probe the physics of the big bang by detecting the energy scale at which inflation occurred if it is above $5\times10^{15}$\,GeV, or place an upper limit if it is below (\S\,\ref{sec:inflation}, Fig.~\ref{fig:clbb})}}&
\multicolumn{1}{l}{\parbox[t]{2in}{Tensor-to-scalar ratio $r$: \\ $\sigma(r) = 1\times10^{-4}$ at $r = 0$; \\ $r < 5 \times 10^{-4}$ at $5\sigma$ confidence level$^a$}} &
\multicolumn{1}{l}{\parbox[t]{2in}{CMB polarization $BB$ power spectrum for modes $2<\ell<300$ to cosmic-variance limit, and CMB lensing power spectrum for modes $2<\ell<1000$ to cosmic-variance limit}}&
\multicolumn{1}{l}{\parbox[t]{2in}{Linear polarization across $62 < \nu < 223$\,GHz over entire sky; foreground separation requires $21 < \nu < 799$\,GHz}}& 
\multicolumn{1}{c}{} &
\multicolumn{1}{l}{\multirow{5}{1.75in}{%
\vskip15pt
Frequency coverage: central frequencies $\nu_c$ from 21 to 799\,GHz
\vskip5pt
Frequency resolution: $\Delta\nu/\nu_c = 25\%$
\vskip5pt
Sensitivity: See Table~\ref{tab:bands}
Combined instrument noise:  $< 0.61\,\mu{\rm K}_{\rm CMB}\sqrt{\rm s}$
\vskip5pt
Angular resolution [for delensing and foreground separation]: ${\rm FWHM} =  6.2' \times ( 155\,{\rm GHz} / \nu_c )$
%\vskip5pt
%Effective aperture: 1.4~m
\vskip5pt
Sampling rate: $( 3 / {\rm Beam FWHM} ) \times ( 336' / {\rm s})$
}}& 
\multicolumn{1}{l}{\parbox[t]{1.5in}{}}& 
\multicolumn{1}{l}{\multirow{7}{1.75in}{%
\vskip10pt
Sun-Earth L2 orbit with Sun-Probe-Earth $< 15^\circ$ (\S\,\ref{sec:mission_design}) 
\vskip5pt
5 yr survey (\S\,\ref{sec:operations})
\vskip5pt
Full sky survey: Spin instrument at 1 rpm; boresight $69^\circ$ off spin axis;
spin axis $26^\circ$ off anti-Sun line, precessing $360^\circ$ / 10hr (\S\,\ref{sec:survey_design})
\vskip5pt
Pointing control: Spin axis $60'$ ($3\sigma$, radial); spin \@ $1 \pm 0.1$ rpm ($3\sigma$) (\S\,\ref{sec:attitude_determination})
\vskip5pt
Pointing stability: Drift of spin axis $< 1'$/1min ($3\sigma$, radial);
jitter $< 20''$/20 ms ($3\sigma$, radial) (\S\,\ref{sec:attitude_determination})
\vskip5pt
Pointing knowledge
(telescope boresight):
$10'' \, (3\sigma$, each axis) from spacecraft attitude;
$1'' \, (1\sigma$, total) final reconstructed (\S\,\ref{sec:attitude_determination})
\vskip5pt
Return and process instrument data:
1.5 Tbits/day (after 4$\times$ compression) (\S\,\ref{sec:ground_segment}, \ref{sec:spacecraft})
\vskip5pt
Thermally isolate instrument from solar radiation and from spacecraft bus (\S\, \ref{sec:radiative_cooling}, \ref{sec:spacecraft})
}}\\
% Line SO2
\noalign{\vskip 1mm}
\cline{2-5}
\noalign{\vskip 1mm}
\multicolumn{1}{l}{}&
\multicolumn{1}{l}{\parbox[t]{2in}{\textbf{SO2}. Probe the physics of the big bang by excluding classes of potentials as the driving force of inflation (\S\,\ref{sec:inflation}, Fig.~\ref{fig:nsr})}}&
\multicolumn{1}{l}{\parbox[t]{2in}{Spectral index ($n_s$) and its derivative ($n_{\rm run}$): $\sigma(n_s) < 0.0015$; $\sigma(n_{\rm run}) < 0.002$}}&
\multicolumn{1}{l}{\parbox[t]{2in}{CMB polarization $BB$ power spectrum for modes $2<\ell<1000$ to cosmic-variance limit}}&
\multicolumn{1}{l}{\multirow{3}{2in}{
\vskip15pt
Intensity and linear polarization across $62 < \nu < 223$\,GHz over the entire sky; foreground separation encompassed by SO1}}& 
\multicolumn{1}{c}{} &
\multicolumn{1}{l}{\parbox[t]{1.75in}{}}& 
\multicolumn{1}{l}{\multirow{7}{1.5in}{%
Frequency coverage: See Tables~\ref{tab:spec_bands} and~\ref{tab:bands}.
\vskip 2pt 
21 bands with $\nu_c$ from 21 to 799\,GHz
\vskip5pt
Frequency resolution: $\Delta\nu/\nu_c = 25\%$
\vskip5pt
Sensitivity: See Table~\ref{tab:bands} %\ref{tab:sensitivity}.
\vskip2pt
Combined instrument noise: $0.43\,\mu{\rm K}_{\rm CMB}\sqrt{\rm s}$
\vskip5pt
Angular resolution: See Table~\ref{tab:spec_bands} and ~\ref{tab:bands}.
\vskip2pt
${\rm FWHM} = 6.2' \times (155\,{\rm GHz} / \nu_c )$;
$1.1'$ for $\nu_c = 799\,$GHz
\vskip5pt
Sampling rate: See Table~\ref{tab:focal_plane} %\ref{Sampling}.
$( 3 / {\rm Beam FWHM}) \times ( 336' / {\rm s})$ 
}}&
\multicolumn{1}{l}{\parbox[t]{1in}{}}\\
% Line SO3
\noalign{\vskip 1mm}
\cline{1-4}
\noalign{\vskip 1mm}
\multicolumn{1}{l}{\multirow{2}{1in}{\vskip5pt \textbf{\textit{Discover how the Universe works: neutrino mass and $N_{\rm eff}$}}}}&
\multicolumn{1}{l}{\parbox[t]{2in}{\textbf{SO3}. Determine the sum of neutrino masses. (\S\,\ref{sec:relics_neutrinos}, Fig.~\ref{fig:DM_baryons})}}&
\multicolumn{1}{l}{\parbox[t]{2in}{Sum of neutrino masses ($\Sigma m_\nu$): $\sigma(\Sigma m_\nu) = 14$\,meV with DESI or Euclid$^b$; independently  $\sigma(\Sigma m_\nu) = 14$\,meV using cluster counts$^c$ }}& %; \comred{$\Sigma m_\nu < ??$\,meV alone}}}&
\multicolumn{1}{l}{\parbox[t]{2in}{CMB polarization power spectra for modes $2<\ell<4000$; CMB intensity maps (to identify clusters using the Compton-$y$ signal)}}&
\multicolumn{1}{l}{\parbox[t]{2in}{}}& 
\multicolumn{1}{c}{} &
\multicolumn{1}{l}{\parbox[t]{1.75in}{}}& 
\multicolumn{1}{l}{\parbox[t]{1.5in}{}}& 
\multicolumn{1}{l}{\parbox[t]{1in}{}}
\\
% Line SO4
\noalign{\vskip 1mm}
\cline{2-4}
\noalign{\vskip 1mm}
&
\multicolumn{1}{l}{\parbox[t]{2in}{\textbf{SO4}. Tightly constrain the thermalized fundamental particle content of the early Universe (\S\,\ref{sec:relics_neutrinos}, Fig.~\ref{fig:Neff_future})}}&
\multicolumn{1}{l}{\parbox[t]{2in}{Number of light relic particle species $N_{\rm eff}$: $ \Delta N_{\rm eff}<0.06\,\, (95\%)$ }}&
\multicolumn{1}{l}{\parbox[t]{2in}{CMB temperature and polarization auto and cross power spectra $2<\ell<4000$}}&
\multicolumn{1}{l}{\parbox[t]{2in}{}}& 
\multicolumn{1}{c}{} &
\multicolumn{1}{l}{\parbox[t]{1.75in}{}}& 
\multicolumn{1}{l}{\parbox[t]{1.5in}{}}& 
\multicolumn{1}{l}{\parbox[t]{1in}{}}
\\
% Line SO5
\noalign{\vskip 1mm}
\cline{1-5}
\noalign{\vskip 1mm}
\multicolumn{1}{l}{\multirow{1}{1in}{\textbf{\textit{Explore how the Universe evolved: reionization}}}}&
\multicolumn{1}{l}{\parbox[t]{2in}{\textbf{SO5}. Distinguish between models that describe the formation of the earliest luminous sources in the Universe (\S\,\ref{sec:extragalacticsci}, Fig.~\ref{fig:ReionizationPICO})}}&
\multicolumn{1}{l}{\parbox[t]{2in}{Optical depth to reionization ($\tau$): $\sigma(\tau) < 0.002$}}&
\multicolumn{1}{l}{\parbox[t]{2in}{CMB polarization $EE$ power spectrum for modes $2<\ell<40$ to cosmic-variance limit}}&
\multicolumn{1}{l}{\parbox[t]{2in}{Linear polarization across $62 < \nu < 223$\,GHz over entire sky; foreground separation encompassed by SO1}}& 
\multicolumn{1}{c}{} &
\multicolumn{1}{l}{\parbox[t]{1.75in}{}}& 
\multicolumn{1}{l}{\parbox[t]{1.5in}{}}& 
\multicolumn{1}{l}{\parbox[t]{1in}{}}
\\
% Line SO6
\noalign{\vskip 1mm}
\cline{1-7}
\noalign{\vskip 1mm}
\multicolumn{1}{l}{\multirow{2}{1in}{{\vskip5pt \textbf{\textit{Explore how the Universe evolved: Galactic structure and dynamics}}}}}&
\multicolumn{1}{l}{\parbox[t]{2in}{\textbf{SO6}. Test models of the composition of Galactic interstellar dust (\S\,\ref{sec:test_composition_models})}}&
\multicolumn{1}{l}{\parbox[t]{2in}{Intrinsic polarization fractions of Galactic dust components to accuracy better than 3\% when averaged over $10'$ pixels }}&
\multicolumn{1}{l}{\parbox[t]{2in}{Spectral energy distribution of interstellar dust polarized emission between 108 and 799\,GHz}}&
\multicolumn{1}{l}{\parbox[t]{2in}{Intensity and linear polarization maps in 12 frequency bands between 108 and 799\,GHz}}& 
\multicolumn{1}{c}{} &
\multicolumn{1}{l}{\parbox[t]{1.75in}{ Encompassed by SO1--5}
}& 
\multicolumn{1}{l}{\parbox[t]{1.5in}{}}& 
\multicolumn{1}{l}{\parbox[t]{1in}{}}
\\
% Line SO7
\noalign{\vskip 1mm}
\cline{2-7}
\noalign{\vskip 1mm}
\multicolumn{1}{l}{}&
\multicolumn{1}{l}{\parbox[t]{2in}{\textbf{SO7}. Determine if magnetic fields are the dominant cause of low Galactic star-formation efficiency (\S\,\ref{sec:magnetic_fields})}}&
\multicolumn{1}{l}{\parbox[t]{2in}{Ratio of cloud mass to maximum mass that can be supported by magnetic field (``Mass to flux ratio'' $\mu$); %\comred{$\sigma(\mu) < ??$};
ratio of gas turbulent energy to magnetic energy (quantified through the Alfv\'{e}n Mach number $\mathcal{M}_A$) on scales 0.05--100\,pc  }}&%\comred{$\sigma(\mathcal{M}_A) < ??$}}}&
\multicolumn{1}{l}{\parbox[t]{2in}{Turbulence power spectrum on scales 0.05--100\,pc; magnetic field strength ($B$) as a function of spatial scale and density; hydrogen column density; gas velocity dispersion$^d$
}}&
\multicolumn{1}{l}{\parbox[t]{2in}{Intensity and linear polarization with $<1$\,pc resolution for thousands of molecular clouds and with $< 0.05$\,pc for the 10 nearest molecular clouds; maps of polarization with 1' resolution over the entire sky}}& 
\multicolumn{1}{c}{} &
\multicolumn{1}{l}{\parbox[t]{1.75in}{
Encompassed by SO1--5, except:
\vskip4pt
Angular resolution: $\le 1.1'$ (at highest frequency)
\vskip4pt
Sensitivity at 799\,GHz: 27.4\, kJy/sr
}}& 
\multicolumn{1}{l}{\parbox[t]{1.5in}}& 
\multicolumn{1}{l}{\parbox[t]{1in}{}}
\\
\noalign{\vskip 1mm}
\hline
\noalign{\vskip 1mm}
\end{tabular}
{\footnotesize
$^a$ The values predicted include delensing and foreground subtraction; see \S~\ref{sec:inflation}. \\
$^b$ Using $\tau$ and the power spectrum of the reconstructed lensing map (\S~\ref{sec:gravitationallensing}), both from PICO's measurements, and baryon acoustic oscillation data from DESI or Euclid. \\
$^c$ The constraint using clusters requires redshifts by future optical and IR surveys. \\
$^d$ Hydrogen column density and gas velocity dispersion will be provided by 21-cm surveys including HI4PI and GALFA-HI. 
}
\end{table}
 
