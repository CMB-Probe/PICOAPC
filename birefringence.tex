The simplest dynamical way to model late-time acceleration of the Universe is with a slowly-evolving 
scalar field---the quintessence~\cite{Carroll:1998zi}. Such a field generically couples to electromagnetism through a Chern-Simons-like 
term, and causes linear polarization of photons propagating cosmological distances to rotate. 
The effect of rotation of polarization is known as cosmic birefringence~\cite{Carroll:1998zi} and it can convert the primordial E mode into B mode in 
the CMB maps. It thus produces parity-violating TB and EB cross-correlations
\cite{Kamionkowski:2008fp,Gluscevic:2009mm} whose magnitude depends on the statistical properties of the rotation field in the sky. 
Even though there is no theoretical prediction for the size of rotation, if observed, it would be evidence for physics beyond the
standard model and a potential probe of dark-energy microphysics. 
Previous studies used quadratic-estimator formalism to constrain uniform and direction-dependent cosmic birefringence~\cite{Gluscevic:2012me}, 
with the best current upper limit coming from sub-degree scale polarization measurements 
with POLARBEAR~\cite{Ade:2015cao}. 
A promising way to pursue search for cosmic birefringence in the future is measurement of the off-diagonal EB cross correlations using PICO maps.

Fig.~\ref{fig:CB-forecast} shows a projection for
\textit{Planck}, the current best limit from POLARBEAR, and a forecast for PICO 155 
GHz channel (polarization white noise: $1.3$ $\mu$K-arcmin, resolution: $6.2'$, full sky coverage).
PICO promises to deliver a large factor of improvement as compared to the the current constraints. 
\begin{figure}[h!]
\centering \includegraphics[width=0.70\textwidth]{images/birefringence-PICO.pdf}
\caption{The projected noise curves for \textit{Planck} and PICO (XX channel) and the current limit from POLARBEAR for the cosmic-birefringence rotation-angle autocorrelation, reconstructed from the EB correlation.}
\label{fig:CB-forecast}
\end{figure}
