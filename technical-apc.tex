\documentclass[PICOAPC.tex]{subfiles}
\newcommand\pdeg{.\!\!\degree}
\newcommand\parcm{.\!\!'}

\section{Technical Overview}
\label{sec:techoverview}

%\label{sec:instrument} %3


%\begin{wrapfigure}{r}{0.6\textwidth}
%\vskip -8pt
%\hfill
%\includegraphics[width=0.6\textwidth]{images/ArchitectureBlockDiagram.png} 
%\vskip -2pt
%\caption{\captiontext
%PICO instrument block diagram. Active coolers provide cooling to the 100\,mK focal-plane, the surrounding 1\,K box, the 4.5\,K secondary reflector, and the 4.5\,K thermal liner that acts as a cold aperture stop. Data from the focal-plane flows to (redundant, cross-strapped) warm readout electronics on the spun module of the spacecraft bus.
%\label{fig:ArchitectureBlockDiagram} }  
%\end{wrapfigure}

PICO will meet all the science-driven instrument requirements %(\S\,\ref{sec:requirements})
with a single instrument: an imaging polarimeter with 21 logarithmically spaced frequency bands centered between 21 and 799\,GHz, each with with a 25\% fractional bandwidth %(Table~\ref{tab:STM})
. The instrument has a passively cooled open Dragone-style telescope (\S\,\ref{sec:telescope} and Fig.~\ref{fig:InstrumentCAD}). The focal plane is populated by arrays of single-moded TES bolometers (\S\,\ref{sec:focal_plane}) and read out using a time-domain multiplexing scheme (\S\,\ref{sec:detector_readout}).  PICO is a closed cycle cryogenic instrument, employing both passive and active cooling stages.  PICO operates from the Earth-Sun L2 and employs a single science observing mode, providing highly redundant coverage of the full sky(\S~\ref{sec:survey_design}). A functional block diagram of the instrument is shown in Figure~\ref{fig:ArchitectureBlockDiagram}. A full description of the reference design can be found in \citep{pico_report}.

\begin{figure}[h] 
\hspace{-0.1in}
\parbox{5.1in}{\centerline{
\includegraphics[width=5.25in]{figures/InstrumentCAD.png} }}
\hspace{0.05in}
\parbox{1.3in}{
\caption{\captiontext
PICO overall configuration in side view and cross section (left), and front view with V-Groove assembly shown semi-transparent (right).  The mission consists of a single science instrument mounted on a structural ring. The ring is supported by bipods on a stage spinning at constant speed relative to a despun module. Figure~\ref{fig:ArchitectureBlockDiagram} shows the functions hosted by each of the modules. 
\label{fig:InstrumentCAD}} }
%\end{center}
\vspace{-0.1in}
\end{figure}


%% begin wcj comment - move to scan section

%During the survey, the instrument is spun at 1~rpm and the spin axis is made to precess about the anti-Sun direction (\S\,\ref{sec:survey_design}). Spacecraft control is simplified by mounting the instrument on a spinning spacecraft module, while a larger non-spinning module houses most spacecraft subsystems (\S\,\ref{sec:spacecraft}). Instrument elements that act as heat sources are accommodated on the spinning module of the spacecraft. Only power and digital data lines cross between the spinning and non-spinning modules.


%% end wcj comment

%\begin{figure}[h]
%\hspace{-0.in}
%\parbox{4.1in}{\centering{
%\includegraphics[width=4.0in]{images/ArchitectureBlockDiagram.png} } }
%\hspace{0.12in}
%\parbox{2.2in}{
%\caption{\captiontext
%PICO instrument block diagram. Active coolers provide cooling to the 100\,mK focal plane, the surrounding 1\,K box, the 4.5\,K secondary reflector, and the 4.5\,K thermal liner that acts as a cold aperture stop. V-grooves provide passive cooling. The instrument, V-grooves, and spacecraft spun module spin together at a rate of 1\,RPM. The spacecraft spun module hosts the 4\,K cooler compressor and drive electronics, the sub-K cooler drive electronics, and the detector warm readout electronics. Only power and digital data lines cross to the spacecraft despun module, which hosts the spacecraft power, telemetry, attitude control, and communication systems (\S\,\ref{sec:spacecraft}).
%\label{fig:ArchitectureBlockDiagram} } } 
%\vspace{-0.in}
%\end{figure}

\subsection{Telescope and Detectors}
\label{sec:telescope} %3.1

The PICO telescope design is driven by a combination of science requirements and physical volume limits. The science requirements are: a large diffraction-limited field of view (DLFOV) sufficient to support approximately $10^4$ detectors; arcminute resolution at 800\,GHz; low spurious polarization; and low sidelobe response. All requirements are met with PICO's 1.4\,m aperture modified open-Dragone design. There are no moving parts in the PICO optical system.The detailed geometric parameterization of the PICO optical design is described by~\citet{Young2018}.


%\begin{figure}%[ht]
%\parbox{4.0in}{
%\includegraphics[width=3.5in]{images/ArchitectureBlockDiagram.png} }
%\hspace{0.1in}
%\parbox{2.3in}{
%\caption{\captiontext
%PICO instrument block diagram. Active coolers provide cooling to the 100\,mK focal-plane, the surrounding 1\,K box, the 4.5\,K secondary reflector, and the 4.5\,K thermal liner that acts as a cold aperture stop. Data from the focal-plane flows to (redundant, cross-strapped) warm readout electronics on the spun module of the spacecraft bus.
%\label{fig:ArchitectureBlockDiagram} }  }
%\end{figure}

%The PICO optical design was selected following a trade study examining cross-Dragone, Gregorian Dragone, and open-Dragone designs~\citep{Young2018}.  The open-Dragone and crossed-Dragone systems offer more diffraction-limited focal-plane area than the Gregorian Dragone one~\citep{deBernardis2018} and are able to support enough detectors to provide the required sensitivity.

%The instrument is configured inside the shadow of a V-groove assembly
%that shields it thermally and optically from the Sun
%(Fig.~\ref{fig:InstrumentCAD} and \S\,\ref{sec:radiative_cooling}).
%The shadow cone depicted in Fig.~\ref{fig:InstrumentCAD} is $29\degree$. The
%angle to the Sun during the survey, $\alpha = 26\degree$
%(\S\,\ref{sec:survey_design} and Fig.~\ref{fig:MissionDesignFigure}),
%is supplemented with a margin of $3\degree$ to account for the radius
%of the Sun ($0\pdeg25$), pointing control error, design margin, and
%alignment tolerances. The open-Dragone design does not require the more massive and voluminous baffles that the cross-Dragone does, and hence can satisfy the aperture size requirement within the shadow cone.

%PICO's initial open-Dragone design~\citep{dragone78, granet2001} has been modified with the addition of an aperture stop and adding corrections to the primary and secondary reflectors to enlarge the DLFOV.

%The primary reflector (270\,cm $\times$ 205\,cm) is passively cooled and the secondary reflector (160\,cm $\times$ 158\,cm) is actively cooled. The highest frequency (900\,GHz) sets the surface accuracy and coating requirement of the reflectors at $\lambda/14 =24\,\mu{\rm m}$. The focal ratio is 1.42. The slightly concave focal surface, which has a radius of curvature of 4.55~m, is telecentric to within $0\pdeg12$ across the entire FOV.

%\comblue{what about the reflector material?}
%An actively cooled aperture stop between the primary and secondary reflectors reduces optical background loading and shields the focal plane from stray radiation. Stray-light analysis of the PICO open-Dragone design using GRASP confirms that the focal plane is protected from direct view of the sky, and that spillover past the primary is suppressed by 80~dB relative to the main lobe for both co-pol and cross-pol beams. Detailed baffle design will be performed during mission formulation.

% \begin{figure}
% \begin{center}
% \includegraphics[width=3in]{figures/OpticsDiagram.png}
% \caption{The optical system is compact.\label{fig:OpticsDiagram}}
% \end{center}
% \end{figure}

%\subsection{Focal Plane}
%\label{sec:focal_plane} %3.2
%
% wrap version of figure 3.3.  Unwrapped version is below.


%PICO's focal plane is populated by an array of \ac{TES} bolometers operating in 21 frequency bands, each with 25\% fractional bandwidth, and band centers ranging from 21 to 799~GHz. 
%The layout of the PICO focal plane is shown in Fig.~\ref{fig:FocalPlaneMechanical} and detailed in Table~\ref{tab:focal_plane}. 
%A conceptual layout of the PICO focal-plane is shown in Fig.~\ref{fig:FocalPlaneMechanical} and detailed in Table~\ref{tab:focal_plane}.

The sensitivity of PICO\rq s detectors is limited by the irreducible backgrounds. Therefore, the required sensitivity determines the detector count in each band. The PICO focal plane has 12,996 detectors, 175~times the number flown aboard \planck , thereby providing the required increase in raw sensitivity
%\begin{wrapfigure}{r}{0.45\textwidth}
%\vskip -8pt
%\hfill
%\includegraphics[width=0.45\textwidth]{figures/FocalPlaneMechanical.png}
%\vskip -4pt
%\caption{\captiontext PICO focal plane. Detectors are fabricated on six types of tiles (shown numbered and colored as in Table~\ref{tab:focal_plane}). The wafers are located on the focal plane such that higher frequency bands, which require better optical performance, are placed nearer to the center. All detectors are within the diffraction-limited performance for their respective frequency bands.  
%\label{fig:FocalPlaneMechanical}}
%\end{wrapfigure}
with a comparably sized telescope. This breakthrough is enabled by development and demonstration in suborbital projects, which now commonly operate arrays of $10^3$--$10^4$ detectors (\S\,\ref{sec:technology_maturation}). %Further technology maturation required for PICO is described in Section~\S\,\ref{sec:technology_maturation}.

%\subsubsection{21--462\,GHz Bands}
%\label{sec:low_freq_det} % 3.2.1
%\vspace{0.02in}

%Several optical-coupling technologies have matured over the past ten years to efficiently use focal-plane area: horns with ortho-mode transducers (OMTs) \citep{Duff2016}; lithographed antenna arrays~\citep{BICEP2015}; and sinuous antennas under lenslets~\citep{Edwards2012}. Horn-coupling and sinuous antenna/lenslet-coupling deliver quantum efficiency $>70\,\%$ over more than an octave of bandwidth, which have been partitioned into two or three colors per pixel. Multi color antenna-arrays have recently been demonstrated in lab testing, and will be deployed on sub-orbital instruments as early as 2019. This coupling enables smaller pixels and therefore they can be more densely packed.
%%%%%%%%
%\input tables/table3.1_wrap.tex
%%%%%%%%

The PICO baseline focal plane employs three-color sinuous antenna/lenslet pixels~\citep{Suzuki2014} for the 21--462\,GHz bands. Niobium microstrips mediate the signals between the antenna and detectors, and partition the %feed's 
wide continuous bandwidth into three narrow channels using integrated, on-wafer, micro-machined filter circuits~\citep{OBrient2013}. Six transition edge sensor bolometers per pixel detect the radiation in two orthogonal polarization states. %The technology maturation required for PICO is described in \$\,\ref{sec:bolometers}.

%\subsubsection{555--799\,GHz Bands}
%\label{sec:high_freq_det} % 3.2.2
%

PICO's highest three frequency channels are beyond the niobium superconducting band-gap, rendering on-wafer, microstrip filters a poor solution for defining the optical passband. For these bands we use feedhorns to couple the radiation to two single-color polarization-sensitive TES bolometers. %Radiation is coupled through horns directly to an absorber in the throat of a waveguide. TES bolometers detect the incident power.  
The waveguide cut-off defines the lower edge of the band, and quasi-optical metal-mesh filters define the upper edge. Numerous experiments have successfully used similar approaches~\citep{Shirokoff2011,Bleem2012,Turner2001}. 
%The technology maturation required for PICO is described in \S\,\ref{sec:dev_arrays}.

%%%%%
%\input tables/table3.2.tex
%%%%%

%\subsubsection{Polarimetry}
%\label{sec:polarimetry} %3.2.3

Polarimetry is achieved by measuring the signals from pairs of two co-pointed bolometers within a pixel that are sensitive to two orthogonal linear polarization states. Half the pixels in the focal plane are sensitive to the $Q$ and half to the $U$ Stokes parameters of the incident radiation, providing sensitivity to the Stokes $I$, $Q$, and $U$ parameters. Two layouts for the distribution of the $Q$ and $U$ pixels on the focal plane have been investigated~\citep{picoweb_QU}; both satisfy mission requirements. %Stokes $I$ is obtained from the sum of the signals of orthogonal detectors.  

%\subsubsection{Sensitivity}
%\label{sec:sensitivity} %3.2.3

%PICO's Current Best Estimate (CBE) sensitivity meets the requirements of the baseline mission with \hbox{$>40\,\%$} margin~\citep{pico_report}. %(Table~\ref{tab:bands}).
%\comblue{this is an odd statement. need to rework}

We developed an end-to-end noise model of the PICO instrument to predict mission sensitivity and provide a metric by which to evaluate mission design trades.
%The model includes four noise sources per bolometer: photon, phonon, Johnson, and readout (from both cold and warm readout electronics). To validate our calculations, we compared two independent software packages that have been validated with several operating CMB instruments. The calculations agreed within 1\% both for individual noise terms and for overall mission noise.
A detailed description of the PICO noise model and its inputs is available in~\citet{Young2018}; small differences between that publication and Table~\ref{tab:bands} are due to refinements of the primary mirror and stop temperatures.
%Laboratory experiments have demonstrated that TES bolometers can be made background-limited in the low loading environment they would experience at L2~\citep{Beyer2012}.
As stated previously, the dominant source of noise derives from the optical background. The sources of optical load include the CMB, and thermal emission of the reflectors, aperture stop, and low-pass filters. The CMB and stop account for at least 50\% of the optical load at all frequencies up to and including 555~GHz. At higher frequencies emission from the primary mirror dominates.
The sensitivity model assumes white statistical noise at all frequencies. Sub-orbital submillimeter experiments have demonstrated TES detectors that are stable to at least as low as 20\,mHz \citep{Rahlin2014}, meeting the requirements for PICO's scan strategy (\S\,\ref{sec:survey_design}). 

\subsection{Detector Readout}
\label{sec:detector_readout} %3.3

Over the past ten years, suborbital experiments have employed voltage-biased TES arrays because their current readout scheme lends itself to superconducting quantum interface device (SQUID)-based multiplexing. Multiplexing reduces the number of wires to the cryogenic stages and thus the total thermal load that the cryocoolers must dissipate. %This approach also simplifies the instrument design.

%In the multiplexing circuitry, SQUIDs function as low-noise amplifiers and cryogenic switches.
The current baseline for PICO is to use a time-domain multiplexer (TDM), which assigns each detector's address in a square matrix of simultaneously read columns, and sequentially cycles through each row of the array~\citep{Henderson2016}. The PICO baseline architecture uses a matrix of 128 rows and 102 columns. The thermal loading on the cold stages from the wire harnesses is subdominant to conductive loading through the mechanical support structures.

Because SQUIDs are sensitive magnetometers, suborbital experiments have developed techniques to shield them from Earth's magnetic field using both highly permeable and superconducting materials~\citep{Hui2018}.  Total suppression factors better than $10^7$ have been demonstrated for dynamic magnetic fields~\citep{Runyan2010}. PICO will use these demonstrated techniques to shield SQUID readout chips from the ambient magnetic environment, which is 20,000 times smaller than near Earth, as well as from fields generated by on-board components, including the 0.1\,K cooler (\S\,\ref{sec:cadr}). This cooler is delivered with its own magnetic shielding, which reduces the field at the distance of the SQUIDs to less than 0.1\,G, which is less than Earth's field experienced by SQUIDs aboard suborbital experiments.
%
SQUIDs are also sensitive to radio-frequency interference (RFI). Several suborbital experiments have demonstrated RFI shielding using aluminized mylar wrapped at cryogenic stages to form a Faraday cage around the SQUIDs~\citep{Kermish2012,EBEX2018,BICEP2014}. Cable shielding extends the Faraday cage to the detector warm readout electronics.

%%begin WCJ note: this would be better in a power budget, or in telemetry
%Redundant warm electronics boxes perform detector readout and instrument housekeeping using commercially available radiation-hardened analog-to-digital converters, requiring 75\,W total. The readout electronics compress the data before delivering them to the spacecraft, requiring an additional 15\,W. PICO detectors produce a total of 6.1\,Tbits/day assuming 16\,bits/sample, sampling rates from Table~\ref{tab:focal_plane}, and bolometer counts from Table~\ref{tab:bands}. \planck \ HFI had a typical 4.7$\times$ compression in flight, with information loss increasing noise by only about $10\,\%$~\citep{Pajot2018,PlanckHFI2011}. Suborbital work has demonstrated 6.2$\times$ lossless compression~\citep{EBEX2017}. PICO assumes 4$\times$ lossless compression.

%%end WCJ note

\medskip
\subsection{Thermal}
\label{sec:thermal} %3.4

%PICO's focal plane is maintained at 0.1\,K to ensure low detector noise while implementing readily available technology~(\S\,\ref{sec:cadr}). To minimize detector noise due to instrument thermal radiation, the aperture stop and reflectors are cooled using both active and radiative cooling (\S\,\ref{sec:4kcooler}, \S\,\ref{sec:radiative_cooling}, Fig.~\ref{fig:ArchitectureBlockDiagram}).

PICO\rq s instrument requirements include cooling power at temperature stages ranging from 40\,K to 0.1\,K.  As with the \planck\ HFI instrument, PICO meets these requirements using a combination of passive and active cooling. The system meets all thermal requirements with robust margins (Table~\ref{tab:cooler}).

\input tables/table3.3.tex

%\smallskip
%\subsubsection{cADR Sub-Kelvin Cooling}
%\label{sec:cadr} %3.4.1

A multi-stage continuous adiabatic demagnetization refrigerator (cADR) maintains the PICO focal plane at 0.1\,K and the surrounding enclosure, filter, and readout components at 1\,K. The cADR employs three refrigerant assemblies operating sequentially to absorb heat from the focal plane at 0.1\,K and reject it to~1\,K. Two additional assemblies, also operating sequentially, absorb this rejected heat at~1\,K, cool other components to 1\,K, and reject heat at~4.5\,K. This configuration provides continuous cooling with small temperature variations at both the 0.1\,K and 1\,K stages.% Heat straps connect the two cADR cold sinks to multiple points on the focal-plane assembly,
%\comor{was 'focal-plane assembly' defined? isn't it at 0.1K?}
%which has high thermal conductance paths built in, to provide spatial temperature uniformity and stability during operation.
%The detector arrays are thermally sunk to the mounting frame.
Heat loads in the range of 30~$\mu$W at 0.1\,K and 1\,mW at 1\,K (time-average) are within the capabilities of current cADRs developed by GSFC (\S~\ref{sec:heritage})~\citep{Shirron2012,Shirron2016} and flown on suborbital balloon flights. The PICO sub-kelvin heat loads are estimated at less than half of this capability (Table~\ref{tab:cooler}).
%\comblue{this paragraph doesn't say anything about the maturity of the multiple stage cADR, and about flight heritage. It should, even if to point to later paragraphs.}

%\subsubsection{The 4.5 K Cooler}
%\label{sec:4kcooler} %3.4.2

%\begin{wrapfigure}{r}{0.4\textwidth}
%\vspace{-8pt}
%\includegraphics[width=2.6in]{figures/CoolerFigure.pdf} 
%\caption{\captiontext
%Projected performance of the NGAS cooler using a multi-stage
%compressor and $^4$He circulating gas~\citep{Rabb2013} meets PICO's requirements
%with $>100\,\%$ margin. PICO requires heat lift of 42\,mW at 4.5\,K (Table~\ref{tab:cooler}). With 250\,W of input power the NGAS cooler is projected to provide 100\,mW of heat lift. We conservatively specify a maximum expected value (MEV) of 350\,W as the compressor's input power, giving 100\,W of additional input power contingency.
%\label{fig:CoolerFigure}} 
%\vspace{-0.15in}
%\end{wrapfigure}
%
A cryocooler system similar to that used on JWST to cool the MIRI detectors~\citep{Durand2008,Rabb2013} backs the cADR and cools the aperture stop and secondary reflector to 4.5\,K. Both Northrop Grumman Aerospace Systems (NGAS, which provided the MIRI coolers) and Ball Aerospace have developed such coolers under the NASA-sponsored Advanced Cryocooler Technology Development Program~\citep{Glaister2006}. NGAS and Ball use slightly different but functionally-equivalent hardware approaches.  The systems utilize Joule-Thomson (J-T) expansion to provide the required cooling power, while using the return flow to intercept conductive parasitic heat loads.
%A 3-stage precooler provides 16\,K precooling to a separate circulated-gas loop. The circulated-gas loop utilizes Joule--Thomson (J-T) expansion, further cooling the gas to 4.5\,K. The J-T expansion point is located close to the cADR heat rejection point and provides to it the lowest temperature. Subsequently, the gas flow intercepts heat conducted to the focal-plane enclosure, then cools the aperture stop and the secondary reflector before returning to the circulation compressor.  %Model-based projections indicate that the coolers delivered for MIRI could meet the PICO 4.5~K heat lift requirement with more than $100\,\%$ margin with these straightforward modifications: replacement of the $^4$He gas used for MIRI's J-T  with $^3$He; and resizing the $^3$He heat exchangers to take advantage of the different gas properties.

%NGAS and Ball are actively working on increasing the flow rate and compression ratio of the J-T compressor,  which should result in higher system efficiency and greater heat-lift relative to the current MIRI cooler.
NGAS uses $^4$He as the circulating gas, as was used for MIRI. Ball uses a somewhat larger compressor and $^3$He as the circulating gas. %Both employ re-optimized heat exchangers.
The NGAS project has completed PDR-level development, and is expected to reach CDR well before PICO begins Phase-A. The projected performance of this cooler is shown in Fig.~\ref{fig:CoolerFigure}; it gives 100~mW at 250\,W input power, which is more than 100\,\% heat lift margin relative to PICO's requirements (Table~\ref{tab:cooler}). For PICO we have assumed an input power of 350\,W.

%The entire precooler assembly and the J-T circulator compressor are located on the warm spacecraft spun module (Fig.~\ref{fig:ArchitectureBlockDiagram}).
%, with relatively short tubing lengths conducting the gas flow from the precooling point to the J-T expansion point.
%All waste heat rejected by the cooler compressors and drive electronics is transferred to the spacecraft heat-rejection system. Unlike JWST, the PICO cooler does not require deployment of the remote cold head.

%\smallskip
%\subsubsection{Radiative Cooling}
%\label{sec:radiative_cooling} %3.4.3

%% WCJ comment
The passive cooling requirements are met with a V-groove assembly consisting of four nested radiation shields
%The V-groove assembly is attached to the bipod struts that support the instrument structural ring. The ring supports the primary reflector and telescope box. The telescope box contains the actively cooled components (\S\,\ref{sec:cadr}, \S\,\ref{sec:4kcooler}), including the secondary reflector, the focal plane and sub-kelvin refrigerator structures. Just inside the box, a thermal liner serves as a cold optical baffle and aperture stop. Instrument integration and test are described in \S\,\ref{sec:iandt}.
%% END WCJ Comment -
%The V-groove assembly consists of four nested radiation shields that provide passive cooling (\S\,\ref{sec:radiative_cooling}). 
%An assembly of four nested V-groove radiators, acting as radiation shields, provides passive cooling
(Fig.~\ref{fig:InstrumentCAD}). This is standard, 30-year old technology~(\S~\ref{sec:heritage}). The outermost shield shadows the interior ones from the Sun. The V-grooves radiate to space, each reaching successively cooler temperatures. The assembly provides a cold radiative environment to the primary reflector, structural ring, and telescope box. As a consequence radiative loads on those elements are smaller than the conductive loads through the mechanical support structures.

\subsection{Instrument Integration and Test}
\label{sec:iandt} % 3.5

The PICO instrument integration and testing plan benefits from heritage and experience with the \planck\ HFI instrument~\citep{Pajot2010}.
Detector wafers are screened prior to selection of flight wafers and focal-plane integration. The cADR and 4\,K cryocooler vendors will qualify those subsystems prior to delivery. The relative alignment of the two reflectors is determined under in-flight thermal conditions using a thermal vacuum (TVAC) chamber and photogrammetry. The flight focal-plane assembly and flight cADR are integrated and tested in a dedicated sub-kelvin cryogenic testbed. The noise, responsivity, and focal-plane temperature stability are characterized using a representative optical load for each frequency band (temperature-controlled blackbody).  The same testbed is used to perform the  polarimetric and spectroscopic calibration.

%PICO screens detector wafer performance prior to selection of flight wafers and focal-plane integration. The cADR and 4\,K cryocooler are qualified prior to delivery. The relative alignment of the two reflectors under thermal contraction is photogrammetrically verified in a thermal vacuum (TVAC) chamber.

%PICO integrates the flight focal-plane assembly and flight cADR in a dedicated sub-kelvin cryogenic testbed. Noise, responsivity, and focal-plane temperature stability are characterized using a representative optical load for each frequency band (temperature-controlled blackbody). Polarimetric and spectroscopic calibration are performed.

The focal plane is integrated with the reflectors and structures, and alignment verified with photogrammetry at cold temperatures in a TVAC chamber.  The completely integrated observatory (instrument and spacecraft bus) is tested in TVAC to measure parasitic optical loading from the instrument, noise, microphonics, and RFI. The observatory is 4.5\,m in diameter and 6.1\,m tall. There are no deployables.

%\bigskip
%\subsection{Design Reference Mission}
%\label{sec:design_reference} %4
%
%\input tables/table4.1_wrap.tex
%
%The PICO design reference mission is summarized in Table~\ref{tab:mission_parameters}.

%\subsection{Concept of Operations}
%\label{sec:operations} %4.1

%The PICO concept of operations is similar to that of the successful \wmap~\citep{Bennett2003} and \planck~\citep{Tauber2010} missions. After launch, PICO cruises to a quasi-halo orbit around the Earth--Sun L2 Lagrange point (\S\,\ref{sec:mission_design}). A two-week decontamination period is followed by instrument cooldown, lasting about two months. After in-orbit checkout is complete, PICO begins its science survey.

%PICO has a single science observing mode, surveying the sky continuously for 5 years using a pre-planned repetitive survey pattern (\S\,\ref{sec:survey_design}). Instrument data are compressed and stored on-board, then returned to Earth in daily 4-hr downlinks, which are concurrent with science observations.
%Ka-band science downlink passes (concurrent with science observations). Because PICO is observing relatively static Galactic, extra-Galactic, and cosmological targets, there are no requirements for time-critical observations or data latency. Presently, there are no plans for targets of opportunity or guest observer programs during the prime mission.
The PICO instrument does not require cryogenic consumables (as the \textit{Planck} mission did), permitting consideration of significant mission extension beyond the prime mission.

%\begin{figure}[!b]
%  \begin{minipage}[b]{0.29\textwidth}
%    \begin{center}
%    \includegraphics[width=1.5in]{figures/InFairing.JPG}
%\caption{\captiontext PICO is compatible with the Falcon~9.\label{fig:InFairing}}
%    \end{center}
%  \end{minipage}
%
%\hfill
%\begin{minipage}[b]{0.67\textwidth}
%    \begin{center}
%    \includegraphics[width=4in]{figures/MissionDesignFigure.png}
%\caption{\captiontext
%  PICO surveys by continuously spinning the instrument about a
%  precessing axis.\label{fig:MissionDesignFigure}}
%   \end{center}
%  \end{minipage}
%\end{figure}

%\subsection{Mission Design and Launch}
%\label{sec:mission_design} %4.1.1

%The science survey is conducted from a quasi-halo orbit around the Earth--Sun L2 Lagrange point. \planck \ and \wmap\ also operated in L2 orbits. L2 orbits provide favorable survey geometry relative to Earth orbits by mitigating viewing restrictions imposed by terrestrial and lunar stray light. The PICO orbit around L2 is small enough to ensure that the Sun--Probe--Earth (SPE) angle is less than $15\degree$. This maintains the telescope boresight $>70\degree$ away from the Earth (Fig.~\ref{fig:MissionDesignFigure}, $70\degree = 180\degree -\alpha - \beta - \rm{SPE}$).
%The orbit provides for a stable thermal environment, and is compatible with the required high data-rate downlink to the Deep Space Network (DSN), which can provide the require 1.5\,Tb/day volume of the compressed data.

%High data-rate downlink to the Deep Space Network (DSN) is available from L2 using near-Earth Ka bands. L2 provides a stable thermal environment, simplifying thermal control. The PICO orbit exhibits no post-launch eclipses.
 
%NASA requires that Probes be compatible with an Evolved Expendable Launch Vehicle (EELV). For the purpose of this study, the Falcon~9~\citep{SpaceX2015} is used as the reference vehicle. Figure~\ref{fig:InFairing} shows PICO configured for launch in a Falcon~9 fairing. The Falcon~9 launch capability for ocean recovery exceeds PICO's 2147\,kg total launch mass (including contingency) by a $50\,\%$ margin.

%Insertion to the halo manifold and associated trajectory correction maneuvers %(TCMs) 
%require 150\,m\,s$^{-1}$ of total $\Delta V$ by the spacecraft. Orbit maintenance requires minimal propellant (statistical $\Delta V\sim 2$\,m\,s$^{-1}$\,year$^{-1}$). The orbital period is $\sim6$\,months. There are no disposal requirements for L2 orbits, but spacecraft are customarily decommissioned to heliocentric orbit.


%\subsubsection{Survey Design}
%\label{sec:survey_design} %4.1.2
 
PICO employs a highly repetitive scan strategy to map the full
sky. During the survey, PICO spins with a period
$T_{\rm spin} = 1$\,min about a spin axis oriented $\alpha=26\degree$
from the anti-solar direction (Fig.~\ref{fig:MissionDesignFigure}). This spin axis
is forced to precess about the anti-solar direction with a period
$T_{\rm prec}= 10$\,hr. The telescope boresight is oriented at an
angle $\beta=69\degree$ away from the spin axis (Fig.~\ref{fig:InstrumentCAD}). This $\beta$ angle is
chosen such that $\alpha + \beta > 90\degree$, enabling mapping of all
ecliptic latitudes. The precession axis tracks along with the Earth in its
yearly orbit around the Sun, so this scan strategy maps the full sky
(all ecliptic longitudes) within 6 months.

PICO's $\alpha=26\degree$ value is chosen to be substantially larger than
the \textit{Planck} mission's $\alpha$ angle ($7.5\degree$) to
mitigate systematic effects by scanning across each sky pixel with a
greater diversity of orientations \citep{Hu2003}. Increasing $\alpha$
further would decrease the Sun-shadowed volume available for the
optics and consequently reduce the telescope aperture size.
PICO's spin-axis precession frequency is 
more than 400 times faster than that of \planck , greatly reducing the effects of any residual $1/f$
noise by spreading the effects more isotropically across pixels.
%A deployable Sunshade was considered, but found not to be required, and was thus excluded in favor of a more conservative and less costly approach.

%The instrument spin rate, selected through a trade study, matches that
%of the \textit{Planck} mission. The study balanced low-frequency
%($1/f$) noise subtraction (improves with spin rate) against
%implementation cost and heritage, pointing reconstruction ability
%(anti-correlated with spin rate), and data volume (linearly correlated
%with spin rate).  The CMB dipole appears in the PICO data timestream
%at the spin frequency (1\,rpm = 16.7\,mHz). Higher multipole signals
%appear at harmonics of the spin frequency, starting at 33\,mHz, above
%the knee in the detector low-frequency noise (\S\,\ref{sec:sensitivity}). A destriping mapmaker applied in data
%post-processing effectively operates as a high-pass filter, as
%demonstrated by \planck~\citep{Kurki-Suonio2009}.


%\subsection{Ground Segment}
%\label{sec:ground_segment} %4.2

%The PICO Mission Operations System (MOS) and Ground Data System (GDS)
%can be built with extensive reuse of standard tools. The PICO concept
%of operations is described in \S\,\ref{sec:operations}.

%%%% There are
%%%% no time critical events, and no driving data latency
%%%% requirements. Routine orbit maintenance activities are required
%%%% roughly every three months (\S\,\ref{sec:mission_design}). The payload
%%%% consists of a single instrument with a single science observing mode
%%%% (a repetitive survey pattern, \S\,\ref{sec:survey_design}).

%All space-ground communications, ranging, and tracking are performed
%by the DSN 34\,m Beam Wave Guide (BWG). X-band is
%used to transmit spacecraft commanding, return engineering data, and
%provide navigation information (S-band is a viable alternative, and
%could be considered in a future trade). Ka-band is used for high-rate
%return of science data.  The baseline 150\,Mb/s transfer rate
%(130\,Mb/s information rate after CCSDS encoding) is an existing DSN
%catalog service~\citep{DSN2015}.  The instrument produces 6.1\,Tb/day,
%which is compressed to 1.5\,Tb/day
%(\S\,\ref{sec:detector_readout}). Daily 4\,hr DSN passes return PICO
%data in 3.1\,hr, with the remaining 0.9\,hr available as needed for
%retransmission or missed-pass recovery.


%\begin{figure}
%\hspace{-0.15in}
%\parbox{5.1in}{\centering{
%\includegraphics[width=5.25in]{figures/Spacecraft.png} } }
%\hspace{0.12in}
%\parbox{1.3in}{
%\caption{\captiontext
%Modular equipment bays provide easy access to all components in the spacecraft de-spun module and enable parallel integration of spacecraft subsystems.\label{fig:Spacecraft}} }
%\vspace{-0.25in}
%\end{figure}

%\subsection{Spacecraft}
%\label{sec:spacecraft} %4.3

%The PICO spacecraft bus is Class~B and designed for a minimum lifetime of 5\,years in the L2 environment. Mission-critical elements are redundant. Flight spares, engineering models, and prototypes appropriate to Class~B are budgeted.

%The aft end of the spacecraft (the ``de-spun module'') is comprised of
%six equipment bays that house standard components
%(Fig.~\ref{fig:Spacecraft}).  The instrument and V-grooves are mounted on
%bipods from the spacecraft ``spun module,'' which contains hosted
%instrument elements (Fig.~\ref{fig:InstrumentCAD}). A motor drives the
%spun module at 1\,rpm to support the science survey requirements
%(\S\,\ref{sec:survey_design}). Reaction wheels on the despun module
%cancel the angular momentum of the spun module and provide three-axis
%control (\S\,\ref{sec:attitude_determination}).

%The bipods that mechanically support the instrument are thermally
%insulating. The passively radiating V-groove assembly thermally
%isolates the instrument from solar radiation and from the bus
%(\S\,\ref{sec:radiative_cooling}). Like \textit{Planck} \citep{Tauber2010}, the V-grooves are
%manufactured using honeycomb material. Additional radiators on the
%spun and despun spacecraft modules ($\sim1$\,m$^2$ each) reject heat
%dissipated by spacecraft subsystems and hosted instrument elements.

%PICO's avionics are dual-string with standard interfaces. Solid-state recorders provide three days of science data storage (4.6 Tbit, \S\,\ref{sec:heritage}), enabling retransmission of missed data (\S\,\ref{sec:ground_segment}).

%PICO employs a fully redundant Ka- and X-band telecommunications
%architecture. The Ka-band system uses a 0.3\,m high-gain antenna to
%support a science data downlink information rate of 130 Mb/s to a
%34\,m BWG DSN ground station with a link margin of 4.8\,dB. The X-band
%system provides command and engineering telemetry communication
%through all mission phases using medium- and low-gain
%antennas. Amplifiers, switches, and all three antennas are on a
%gimballed platform, enabling Ka and X-band downlink concurrent with
%science observations.

% The heritage power electronics are dual-string.
%A 74\,A-hr Li-ion battery is sized for a 3\,hr launch phase with 44\,\% depth of discharge.
%After the launch phase, the driving
%mode is telecom concurrent with science survey (1320\,W including 43\,\% contingency).
%Solar cells on the aft side of the bus (5.8\,m$^2$ array, $\alpha=26\degree$ off-Sun) support this mode with positive power,  and unused area in the solar array plane (7.4\,m$^2$ more area by growing to 4.5\,m diameter) affords 125\,\% margin
%(Fig.~\ref{fig:Spacecraft}).

%The propulsion design is a simple mono-propellant blow-down hydrazine
%system with standard redundancy. Two aft-pointed 22\,N thrusters
%provide $\Delta V$ and attitude control for orbit insertion and
%maintenance (\S\,\ref{sec:mission_design}), requiring 140 kg of
%propellant.  Eight 4\,N thrusters provide reaction-wheel momentum
%management and backup attitude-control authority (60\,kg of
%propellant). Accounting for ullage (14\,kg), the baseline propellant
%tank fill fraction is 77\,\%.


%\subsubsection{Attitude Determination and Control}
%\label{sec:attitude_determination} %4.3.1

%PICO uses a zero net angular momentum control architecture with heritage from the SMAP mission (\S\,\ref{sec:heritage}). PICO's instrument spin rate (1\,rpm) matches that of the \planck\ mission, but the precession of the spin axis is faster (10\,hr vs 6\, months), and the precession angle larger ($26\degree$ vs $7.5\degree$). These differences make the spin-stabilized \planck \ control architecture impractical. 
% because of the amount of torque that would be required to drive precession.

%The PICO instrument spin rate is achieved and maintained using a spin motor. The spin motor drive electronics provide the coarse spin rate knowledge used for controlling the spin rate to meet the $\pm0.1$\,rpm requirement. Data and power are passed across the interface using slip rings.

%PICO requires $220$\,N\,m\,s to cancel the angular momentum of the instrument and spacecraft spun module at 1\,RPM. This value includes mass contingency and is based on the CAD model. Three Honeywell HR-16 reaction wheel assemblies (RWAs), each capable of 150\,N\,m\,s, are mounted on the despun module parallel to the instrument spin axis, and spin opposite to the instrument to achieve zero net angular momentum. The despun module is three-axis stabilized. The spin axis is precessed using three RWAs mounted normal to the spin axis in a triangle configuration. Each set of three RWAs is sized such that two could perform the required function with margin, providing single fault tolerance.

%Spin-axis pointing and spin-rate knowledge are achieved and maintained using star tracker and inertial measurement unit (IMU) data. The attitude determination system is single-fault tolerant, with two IMUs each on the spun and despun modules, and two star trackers each on the spun and despun modules. Two Sun sensors on the despun module are used for safe-mode contingencies and instrument Sun avoidance. All attitude control and reconstruction requirements are met, including spin axis control $< 60$\,arcmin with $< 1$\,arcmin/min stability, and reconstructed pointing knowledge $< 10$\,arcsec (each axis, $3\sigma$).

%Additional pointing reconstruction is performed in post-processing using the science data.  The PICO instrument will observe planets (compact, bright sources) nearly every day.  By fitting the telescope pointing to the known planetary ephemerides, the knowledge of the telescope boresight pointing and the relative pointing of each detector will improve to better than 1\,arcsec (each axis, $3\sigma$). \planck , with fewer detectors, making lower \ac{SNR} measurements of the planets, and observing with a scan strategy that acquired measurements of each planet only once every 6\,months, demonstrated 0.8\,arcsec ($1\sigma$) pointing reconstruction uncertainty in-scan and 1.9\,arcsec ($1\sigma$) cross-scan~\citep{2016A&A...594A...1P}.
%(Planck, Planck 2015 results. I. Overview of products and scientific results 2016)


%%% WCJ note: I don't see a need for this as a standalone section in the APC
%%%
%\bigskip
%\newpage
\subsection{Technology Drivers}
%\section{Technology Drivers}
%\label{sec:technology_maturation} %5
%
%\begin{wrapfigure}[10]{r}{3.80in}  % r is right aligned, l is left. Capital letters allow figure to float on page.
%\vspace{-15pt} % move figure up to align with section headings.
%\parbox{2.35in}{
%\includegraphics[width=2.35in]{figures/SPT3G.jpg} }  %% image was 3 in wide in past version.
%\parbox{1.40in}{
%\caption{\captiontext SPT-3G operates a focal plane with sinuous antenna-coupled, three-band pixels with 16,000 bolometers~\citep{Dutcher2018}. Each pixel couples radiation to bands at 95, 150, and 220~GHz.\label{fig:spt_fp} }
%}
%\end{wrapfigure}

PICO builds off of the heritage of \planck-HFI and \textit{Herschel}.  Since the time of \planck\ and \textit{Herschel}, suborbital experiments have used monolithically fabricated TES bolometers and multiplexing schemes to field instruments with thousands of \ac{TES} bolometers per camera (Fig.~\ref{fig:spt_fp}). %By the time PICO enters Phase~A, S3 experiments plan to be operating nearly 100,000 \ac{TES} bolometers in several independent cameras~\citep{Simons2018,biceparray,spt3g}.

 The remaining technology developments required to enable the PICO baseline design are:
\begin{enumerate}
\item extension of three-color antenna-coupled bolometers down to 21\,GHz and up to 462\,GHz (\S\,\ref{sec:bolometers});
\item construction of high-frequency direct absorbing arrays and laboratory testing (\S\,\ref{sec:dev_arrays});
\item beam line and 100\,mK testing to simulate the cosmic ray environment at L2 (\S\,\ref{sec:env_testing});
\item expansion of time-division multiplexing to support 128 switched rows per readout column (\S\,\ref{sec:multiplexing}).
%\item Simulation software (\S\,\ref{sec:simulation}). {\color{red}Shaul to fill this in.}
\end{enumerate}
All of these developments are straightforward extensions of technologies already available today, with heritage on suborbital missions (See Table~\ref{tab:suborbital}.  We recommend continued support to complete development of these technologies through the milestones described in Table~\ref{tab:technologies}.  A detailed discussion of these technical challenges is included in the full PICO report \citep{pico_report}

%\medskip
%\subsection{21--462\,GHz Bands}
%\label{sec:bolometers} %5.1

%
%Suborbital teams have successfully demonstrated a variety of optical-coupling schemes, including horns with ortho-mode transducers (OMTs), lithographed antenna arrays, and sinuous antennas under lenslets (Table~\ref{tab:suborbital1}). All have achieved background-limited performance with sufficient margin on design parameters to achieve this performance in the lower background environment at L2. All have been packaged into modules and focal-plane units in working cameras representative of the PICO integration. Experiments have already used a number of PICO's observing bands between 27\,GHz and 270\,GHz (Table~\ref{tab:suborbital1}).  To date, statistical map depths of 3\,$\mu$K$_{\rm CMB}$\,arcmin have been achieved over small sky areas, which is within a factor of five of PICO's CBE over the entire sky (Table~\ref{tab:bands}). %and have demonstrated systematic control better than this level through full-pipeline simulations and null-test analysis (jackknife tests).

% Other experiments have
% successfully deployed two-color pixels. All of these detector arrays
% have been packaged into modules and focal-plane units in working
% cameras representative of the PICO integration.

\input tables/table5.1.tex
%\costfootnote

%The baseline PICO instrument requires three-color dual-polarized antenna-coupled bolometers covering bands from 21 to 462\,GHz (\S\,\ref{sec:low_freq_det}).  The sinuous antenna has the bandwidth to service three bands per pixel, whereas horns and antenna arrays have only been used for two. Our baseline is to use a three-band sinuous antenna, although we have designs that use two- or one-band per pixel and have the same or similar baseline noise as PICO~(\S~\ref{sec:technology_descopes}). SPT-3G has used the PICO-baselined three-color pixel design to deploy 16,000 detectors covering 90/150/220\,GHz~\citep{Dutcher2018}.


%The extension to lower frequencies requires larger antennas and therefore control of film properties and lithography over larger areas. Scaling to higher frequencies requires tighter fabrication tolerances and electromagnetic wave transmission losses tend to increase due to material properties. Current anti-reflection technologies for the lenslets need to be extended with thicker and thinner layers to cover the lowest and highest frequency channels. These developments will require control of cleanliness and understanding of process parameters. Changes to elements in the light path will require characterization of beam properties.

\input tables/table5.2.tex

%The direction of polarization sensitivity of the sinuous antenna varies with frequency, thus presenting a potential source of systematic error. Over 25\% bandwidth, the variation is approximately $\pm 5$~deg~\citep{obrient2008b}. There are solutions to this in the focal-plane design, measurements, data analysis, and free parameters of the sinuous antenna geometry.  A recent study found that pre-flight characterization of the effect through measurements can readily mitigate it as a source of systematic uncertainty~\citep{picoweb_wobble}. Studies with current field demonstrations, such as with the data of SPT-3G, will be particularly important. The PICO concept is robust to any challenges in developing three-color pixels; \S\,\ref{sec:technology_descopes} describes options to descope to two- and one-color pixels, technologies for which the polarization sensitivity is constant as a function of frequency.

%\medskip
%\subsection{555--799\,GHz bands}
%\label{sec:dev_arrays}

%The baseline PICO instrument requires single-color, horn-coupled, dual-polarization, direct-absorbing bolometers from 555 to 799\,GHz (\S\,\ref{sec:high_freq_det}).  \planck\ and \textit{Herschel} demonstrated the architecture of horns coupled to direct absorbing bolometers. 
%(Fig.~\ref{fig:DirectAbsorbing})    
%Ground experiments with similar designs have deployed focal planes with hundreds of horn-coupled spiderweb bolometers, replacing the \textit{Planck} and \textit{Herschel} NTD-Ge thermistors with TESs, and adjusting time constants as necessary (Table~\ref{tab:suborbital2}). \planck -HFI, SPT-pol, and BICEP demonstrated dual-polarized detectors. \textit{Herschel} and SPT-SZ demonstrated monolithic unpolarized detectors. PICO will require detectors that merge these two designs in monolithic dual-polarized arrays. Since all the components of the technology already exist, the remaining necessary development is the packaging. Filled arrays of detectors such as Backshort Under Ground (BUG) bolometers are also an option~\citep{Staguhn2006}.

\input tables/table5.3.tex

%The greatest remaining challenge is the low risk development of a packaging design.

%\subsection{Environmental Testing}
%\label{sec:env_testing}

%\begin{wrapfigure}[27]{r}{0.35\textwidth}  % r is right aligned, l is left. Capital letters allow figure to float on page.
%\centering
%\vspace{-5pt} % move figure up to align with section headings.
%\includegraphics[width=0.35\textwidth]{figures/DirectAbsorbing.png}  %% figure was 2.5 in, .4 /textwidth is 2.6in.
%\vspace{-0.25in}
%\caption{\captiontext Top: \planck\ used horns to couple the electromagnetic radiation to its detectors. Horn coupling has been used in other experiments, and is the baseline for PICO's coupling between 555 and 799~GHz. Bottom: The photograph shows a dual-polarization, direct-absorbing bolometer from BICEP. The technology was also used with SPT-Pol and \planck-HFI for 143--343\,GHz bands.
%\label{fig:DirectAbsorbing} }
%\end{wrapfigure}

%Laboratory tests and in-flight data from balloons suggest that TES
%bolometer arrays may be more naturally robust against cosmic rays than
%the individual NTD-Ge bolometers used in \textit{Planck}. PICO will leverage lessons
%learned from \textit{Planck} and ensure robust thermal sinking of
%detector array substrates. Cosmic-ray
%glitches have fast recovery times and low coincidence rates
%\citep{SPIDER2018,Filippini_inprep}. Residual risk can be retired with 100\,mK
%testing where the array heat sinking may be weaker, and beam-line
%tests to simulate the expected flight environment.

%\subsection{Multiplexing}
%\label{sec:multiplexing}

%More than ten experiments have used time-domain multiplexer (TDM) readout. SCUBA2 on JCMT has 10,000 pixels, nearly as many detectors as planned for PICO~\citep{Holland2013}. Most of these experiments have used 32-row multiplexing. Recently ACT has expanded this to 64-row multiplexing~\citep{Henderson2016}.

%PICO's sensitivity requirements dictate the use of 13,000 transition-edge-sensor bolometers and a multiplexed system.  Our baseline design is to use TDM readout with 128 switched rows per readout column (TDM-128$\times$). The leap to TDM-128$\times$ requires: \\
%$\bullet$ development of fast-switched room temperature electronics; and \\
%$\bullet$ system engineering of room temperature to cryogenic row-select cabling to ensure sufficiently fast row-switch settling times.

%The historical row revisit rate for bolometric instruments using 32$\times$ TDM has been 25\,kHz \cite[e.g.,][]{BICEP2015}. However, X-ray instruments using TDM routinely switch between rows at 6.25~MHz~\citep{Doriese2016}. The PICO baseline assumes a 6.25~MHz switch rate and TDM-128$\times$, which dictates a row-revisit rate %(effective sampling rate) 
%of 48.8\,kHz. To limit aliased noise, PICO implements low-pass filters in each readout channel with a bandwidth of 6\,kHz, dictated by detector stability considerations and the required $\sim1$\,kHz signal bandwidth.  With these parameters and using the same TDM multiplexer SQUID design, the increased total noise due to aliasing is less than 15\,\% and is included in our detector noise budget.  The system engineering study will culminate in a demonstration of TDM-128$\times$ SQUID aliased noise below PICO detector sensitivity requirements.


%\subsection{Technology Descopes}
%\label{sec:technology_descopes} %5.3

%A descope from three-color sinuous antenna/lenslet-coupled pixels to two-color horn-coupled, or to single color antenna-array pixels remains a viable alternative should the three-color technology not mature as planned. In both alternative options, bands above 555~GHz are the same as the baseline. For the lower frequencies, the two-color horn-coupled pixel option contains 8,840 detectors and has 19 colors. Because horns have a $2.3:1$ bandwidth, each of the two bands in a pixel has 35\,\% bandwidth (compared to the baseline 25\,\%), which compensates for pixel count, resulting in 0.61\,$\mu$K$_{\rm CMB}$\,arcmin aggregate CBE map depth. This is the same as the three-color CBE map depth, and affords the same $40\,\%$ margin relative to the 0.87\,$\mu$K$_{\rm CMB}$\,arcmin baseline requirement (Table~\ref{tab:bands}). Detailed analysis would be performed to assess the impact of the coarser spectral resolution on signal component separation. Single color antenna-array pixels can have higher packing density than the other two architectures. This option has 6,540 detectors, 21 colors, each with 30\,\% bandwidth, and a noise level of 0.74\,$\mu$K$_{\rm CMB}$\,arcmin, leaving only 17\% noise margin relative to the requirement. 

%A descope from three-color sinuous antenna/lenslet-coupled pixels to two-color horn-coupled pixels remain a viable alternative should the three-color technology not mature as planned. We have a design for a PICO-size focal-plane with two-color horn-coupled pixels at the lower frequencies and the baseline one-color pixels at the higher frequencies. It contains 8,840 detectors (compared to the baseline with 12,966) and has 19 colors (baseline has 21 colors). Because horns have a $2.3:1$ bandwidth, each of the two bands in a pixel has 35\,\% bandwidth (compared to the baseline 25\,\%), which compensates for pixel count, resulting in 0.61\,$\mu$K$_{\rm CMB}$\,arcmin aggregate CBE map depth, which is the same as the three-color CBE map depth, and affords the same $40\,\%$ margin relative to the 0.87\,$\mu$K$_{\rm CMB}$\,arcmin baseline requirement (Table~\ref{tab:bands}). Detailed analysis would be performed to assess the impact of the coarser spectral resolution on signal component separation (\S\,\ref{sec:signal_separation}).

%\subsection{Enhancing Technologies}
%\label{sec:enhancing_technologies} %5.4

%The following technologies are neither required nor assumed by the PICO baseline concept. However, they represent opportunities to extend scientific capabilities or simplify engineering.

%PICO baselines TDM readout because of its relative maturity and demonstrated sensitivity and stability in relevant science missions. Lab tests of frequency-domain multiplexing (FDM) give comparable performance with higher multiplexing factors and lower thermal loads on cryogenic stages relative to TDM, but with higher ambient temperature power consumption. Suborbital experiments such as SPT-3G are using FDM to read out focal planes comparable in size to PICO.

%Microwave frequency SQUID multiplexing can increase the multiplexing density and reduce the number of wires between the 4\,K and ambient temperature stages~\citep{Dober2017,Irwin2004}. Kinetic inductance detectors and Thermal KIDs can further reduce the wire count, obviate the need for SQUID-based amplifiers, and simplify integration by integrating the multiplexing function on the same substrate as the detectors~\citep{McCarrick2018,Steinbach2018,Johnson2018}. The cost to develop these technologies is \$3--4M/year, with a high chance of reaching TRL-5 before Phase~A.
%\costfootnote

%\end{document}
